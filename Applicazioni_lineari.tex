% APPLICAZIONI LINEARI

\begin{definition}[Spazio delle funzioni]
	Sia $ A $ e $ V $ un $ \K $-spazio vettoriale. L'insieme $ V^A = \mathscr{F}{(A, V)} = \{f \colon A \to V\} $ con le operazioni di
	\begin{itemize}
		\item \emph{somma} $ \forall f, g \in \mathscr{F}(A, V) $: $ \forall x \in A, (f + g)(x) = f(x) + g(x) $;
		\item \emph{prodotto per scalare} $ \forall f \in \mathscr{F}(A, V), \forall \alpha \in \K $: $ \forall x \in A, (\alpha \cdot f)(x) = \alpha \cdot f(x) $
	\end{itemize} 
	è lo spazio vettoriale delle funzioni da $ A $ in $ V $. 
\end{definition}

\begin{definition}[Applicazione lineare]
	Siano $ V $ e $ W $ due $ \K $-spazi vettoriali. Diremo che una funzione $ L \colon V \to W  $ è un'applicazione lineare (o mappa lineare o omomorfismo) se $ L $ soddisfa le seguenti proprietà
	\begin{enumerate}[label = (\roman*)]
		\item $ \forall v_1, v_2 \in V $ vale $ L(v_1 + v_2) = L(v_1) + L(v_2) $
		\item $ \forall v \in V, \forall \lambda \in \K $ vale $ L(\lambda v) = \lambda L(v)$
	\end{enumerate}
	Diretta conseguenza è la seguente proprietà chiave delle applicazioni lineari
	\begin{enumerate}[resume, label = (\roman*)]
		\item $ L(O_V) = O_W $.
	\end{enumerate}
	L'insieme delle applicazioni lineari $ \mathscr{L}{(V, W)} = \{L \colon V \to W : L \text{ è lineare}\} $ è un sottospazio vettoriale di $ \mathscr{F}{(V, W)} $
\end{definition}

\begin{definition}[Nucleo]
	Siano $ V $ e $ W $ due spazi vettoriali su un campo $ \K $ e sia $ L \colon V \to W  $ un'applicazione lineare. Definiamo nucleo o kernel di $ L $, e scriveremo $ \ker{L}$, l'insieme degli elementi di $ V $ la cui immagine attraverso $ L $ è lo zero di $ W $. Formalmente \[\ker{L} = \{v \in V \colon L(v) = O_W\} \]
\end{definition}

\begin{prop}[di nucleo e immagine]
	Valgono le seguenti proprietà:
	\begin{enumerate}[label = (\roman*)]
		\item Sia $ L \in \mathscr{L}{(V, W)} $. Allora $ \ker{L} $ è un sottospazio vettoriale di $ V $.
		\item Sia $ L \in \mathscr{L}{(V, W)} $. Allora $ \im{L} $ è un sottospazio vettoriale di $ W $.
			Più in generale se $U$ è un sottospazio di $V$ allora $ \im{L|_U} $ è un sottospazio di $W$.
	\end{enumerate}
\end{prop}

\begin{prop}
	$ L \in \mathscr{L}{(V, W)} $ è iniettiva se e solo se $ \ker{L} = \{O_V\} $.
\end{prop}

\begin{prop}[Composizione di applicazioni lineari]
	La composizione di due applicazioni lineari è ancora un'applicazione lineare.
\end{prop}

\begin{thm}[Inversa di un'applicazione lineare]
	L'inversa di una applicazioni lineare (se esiste) è ancora un'applicazione lineare
\end{thm}

\begin{thm}
	Sia $ L \colon \K^n \to \K^n $ un'applicazione lineare tale che $ \ker{L} = \{O_V\} $. Se $ v_1, \ldots, v_n \in V $ sono vettori linearmente indipendenti, anche $ L(v_1), \ldots, L(v_n) $ sono vettori linearmente indipendenti di $ W $.
\end{thm}

\begin{thm}[delle dimensioni]
	Siano $ V $ e $ W $ spazi vettoriali su un campo $ \K $ e sia $ L \colon V \to W $  un'applicazione lineare. Allora vale \[\dim{V} = \dim{\im{L}} + \dim{\ker{L}}\]
\end{thm}

\begin{corollary}
	Valgono le seguenti relazioni:\\
	contenuto...
\end{corollary}

\begin{thm}[Biettività applicazioni lineari]
	Sia $ L \colon \K^n \to \K^n $ un'applicazione lineare. Se $ \ker{L} = \{O_V\} $ e $ \im{L} = W $, allora $ L $ è biettiva e dunque invertibile
\end{thm}

\begin{definition}[Isomorfismo]
	$ L \in \mathscr{L}{(V, W)} $ biettiva si dice isomorfismo. Se tale applicazioni esiste si dice che $ V $ e $ W $ sono isomorfi.
\end{definition}

\begin{definition}[Endomorfismo]
	$ L \in \mathscr{L}{(V, V)} $ dallo spazio in sé si dice endomorfismo. L'insieme degli endomorfismi viene indicato con $ \End{(V)} $ ed è un sottospazio vettoriale di $ \mathscr{F}{(V, V)} $
\end{definition}

\begin{prop}[Invertibilità endomorfismi]
	Un'endomorfismo è invertibile se e solo se è iniettivo.
\end{prop}

\begin{thm}[Applicazioni lineari e basi 1]
	Sia $V$ uno spazio vettoriale e $ \mathscr{B} = \{v_1, \ldots, v_n\} $ una base. Sia $ L \in \mathscr{L}{(V, W)} $ un'applixazione lineare. Allora $L$ è determinata dalla conoscenza di $L(v_i)$ per tutti gli elementi della base.
\end{thm}

\begin{thm}[Applicazioni lineari e basi 2]
	Siano $V$ e $W$ due spazi vettoriali, $ \mathscr{B} = \{v_1, \ldots, v_n\} $ una base di $V$ e $w_1, \ldots, w_n$ dei vettori di $W$. Esiste un'unica applicazione lineare $L \in \mathscr{L}{(V, W)}$ tale che $L(v_i) = w_i$.
\end{thm}

% Altri fatti di vaga utilità

\begin{fatto}[decomposizione di Fitting]
	Sia $ V $ un $ \K $-spazio vettoriale di $ \dim{V} = n $ e $ f \in \End{(V)} $. Allora esiste un intero $ k \leq n $ tale che \begin{enumerate}[label = (\roman*)]
		\item $ \ker{f^{k}} = \ker{f^{k + 1}} $;
		\item $ \im{f^{k}} = \im{f^{k + 1}} $;
		\item $ f\lvert_{\im{f^{k}}} \colon \im{f^{k}} \to \im{f^{k}} $ è un isomorfismo;
		\item $ f(\ker{f^{k}}) \subseteq \ker{f^{k}} $;
		\item $ f\lvert_{\ker{f^{k}}} \colon \ker{f^{k}} \to \ker{f^{k}} $ è nilpotente;
		\item $ V = \ker{f^{k}} \oplus \im{f^{k}} $.
	\end{enumerate}
\end{fatto}

\begin{fatto}
	Sia $ V $ un $ \K $-spazio vettoriale di $ \dim{V} = n $ e $ f \in \End{(V)} $. Allora:
	\begin{enumerate}
		\item $ \forall j \in \N, \ \ker{f^{j}} \subseteq \ker{f^{j + 1}} $;
		\item se esiste $ j \in \N : \ker{f^{j}} = \ker{f^{j + 1}} $ allora $ \forall m \geq j, \ \ker{f^{m}} = \ker{f^{m + 1}} $;
		\item se esiste $ j \in \N : f^{j} = 0 $ (endomorfismo nullo), allora $ f^{n} = 0 $. 
	\end{enumerate}
\end{fatto}