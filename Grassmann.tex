\begin{thm}[formula di Grassmann]
	Siano $ A $ e $ B $ sottospazi vettoriali di $ V $ su un campo $ \K $. Vale \[\dim{A} + \dim{B} = \dim{(A + B)} + \dim{(A \cap B)}\]
\end{thm}

\begin{definition}[somma diretta]
	Dati $ A $ e $ B $ sottospazi di $ V $ su un campo $ \K $, si dice che $ A $ e $ B $ sono in somma diretta se $ A \cap B = \{O_V\} $.\\
	In modo del tutto equivalente $ A $ e $ B $ sono in somma diretta se e solo se $ \dim{A} + \dim{B} =  \dim{(A + B)} $.	
\end{definition}

\begin{definition}[somma diretta di $ k $ sottospazi]
	$ U_1, \ldots , U_k $ sottospazi di $ V $ su un campo $ \K $ si dicono essere insomma diretta se $ \forall i \in \{1, \ldots, k\} $ vale \[U_i \cap (U_1 + \ldots + \hat{U}_i + \ldots + U_k) = \{O_V\}\]
	In modo equivalente $ U_1, \ldots , U_k $ sono insomma diretta se e solo se \[\dim{U_1} + \ldots + \dim{U_k} = \dim{(U_1 + \ldots + U_k)}\]
\end{definition}

\begin{definition}[complementare di un sottospazio]
	Sia $ A $ un sottospazio di $ V $ su un campo $ \K $. Un complementare di $ A $ è un sottospazio $ B $ di $ V $ tale che
	\begin{enumerate}[label=(\roman*)]
		\item $ A \cap B = \{O_V\} $ ($ A $ e $ B $ sono in somma diretta)
		\item $ A + B = V $
	\end{enumerate}
	In tal caso scriveremo che $ A \oplus B = V $. 
\end{definition}