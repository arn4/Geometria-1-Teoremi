% MATRICI

\begin{definition}[Matrice]
	Una matrice $ m \times n $ a coefficienti in $ \K $ è un tabella ordinata di $ m $ righe e $ n $ colonne i cui elementi appartengono ad un campo $ \K $. L'insieme delle matrici $ m \times n $ a coefficienti nel campo $ \K $ viene indicato con $ \mathrm{Mat}_{m \times n}{(K)} $ ed è uno spazio vettoriale. \\ Dati $ a_{ij} \in \K $ con $ i = 1, \ldots, m $ e $ j = 1, \ldots, n $ diremo che $ A \in \mathrm{Mat}_{m \times n} (\K) $ e scriveremo
	\[ A = (a_{ij}) =
	\begin{pmatrix}
		a_{11} & \cdots  & a_{1n} \\
		\vdots & \ddots & \vdots \\
		a_{m1} & \cdots  & a_{mn} \\
	\end{pmatrix}\]
\end{definition}

\begin{definition}[Matrice diagonale e Identità]
	contenuto...
\end{definition}

\begin{definition}[Matrice trasposta]
	contenuto...
\end{definition}

\begin{propriety}[della trasposta]
	contenuto...
\end{propriety}

\begin{definition}[Prodotto tra matrici]
	contenuto...
\end{definition}

\begin{propriety}[del prodotto tra matrici]
	Il prodotto tra matrici gode delle seguenti proprietà:
	\begin{enumerate}[label=(\roman*)]
		\item $ A(B+C) = AB + AC $;
		\item $ (\alpha A)B = A(\alpha B) = \alpha AB $;
		\item $ (AB)C = A(BC) $;
		\item $ ^t(AB) = ^tB^tA $; %TODO: sistemare i trasposti
		\item $(AB)^{-1} = B^{-1}A^{-1}.$
	\end{enumerate}
\end{propriety}

\begin{definition}[Matrice inversa]
	Si chiama inversa di una matrice quadrata $A$ e si indica con $A^{-1}$ la matrice tale che
	\[AA^{-1}=A^{-1}A=Id.\]	
\end{definition}

% SPOSTATO
%\begin{definition}[Matrici coniugate]
%	Due matrici $ A $ e $ B $ si dicono coinugate se esiste una matrice $ P $ invertibile tale che %	\[B = P^{-1} A P.\] Matrici coniugate rappresentano la stessa applicazioni lineari viste in due basi diverse. 
%\end{definition}

\begin{definition}[Traccia]
	Sia $ M $ una matrice quadrata $ n \times n $. La traccia di $ M $ è la somma degli elementi sulla diagonale 
	\[\tr{(M)} = \tr{ 
	\begin{pmatrix}
	a_{11} & \cdots  & a_{1n} \\
	\vdots & \ddots & \vdots \\
	a_{n1} & \cdots  & a_{nn} \\
	\end{pmatrix}}
	= a_{11} + \ldots + a_{nn}\]
\end{definition}

\begin{propriety}[della traccia]
	La traccia gode delle seguenti proprietà
	\begin{enumerate}[label=(\roman*)]
		\item $ \tr (A + B) = \tr(A) + \tr(B) $ e $ \tr(\lambda A) = \lambda \, \tr (A) $
		\item $ \tr(\prescript{t}{}{A}) = \tr (A) $
		\item $ \tr (AB) = \tr (BA) $. Più in generale una permutazione ciclica del prodotto non cambia la traccia.
		\item la traccia è inariante per coniugio
			\footnote{Viene definito in seguito il coniugio};
	\end{enumerate}
\end{propriety}

\begin{definition}[Rango]
	Si dice rango di una matrice la dimensione dello Span dei vettori colonna.
\end{definition}

\begin{definition}[Riduzione per righe]
	contenuto...
\end{definition}

\begin{definition}[Riduzione per colonne]
	contenuto...
\end{definition}

\begin{definition}[Pivot]
	contenuto...
\end{definition}

\begin{thm}
	Sia $A$ una matrice, vale la seguente uguaglianza:
	\[ \rg(A) = \#\text{pivot} = \#\text{(righe $\ne 0$)} = \dim \left ( Span (\text{vettori riga} ) \right ) \]
\end{thm}

\begin{corollary}
	Per ogni matrice $A$ si ha che \[ \rg(A) = \rg(^t A).\]
\end{corollary}

\begin{prop}
	Le righe non nulle di una matrice ridotta per righe sono linearmente indipendenti e le colonne che contengono pivot sono linearmente indipendenti. \\
	Analogamente vale per la riduzione per colonne. 
\end{prop}

\begin{fatto}
	Tutte e sole le matrici $ A \in \mathrm{Mat}_{n \times n}(\K) $ che commutano con ogni matrice $ B \in  \mathrm{Mat}_{n \times n}(\K) $ sono multipli dell'identità. \[AB = BA, \ \forall B \in \mathrm{Mat}_{n \times n}(\K) \quad \iff \quad \exists \lambda \in \K : A = \lambda \Id\]
\end{fatto}

