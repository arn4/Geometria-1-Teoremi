% MATRICI

\begin{definition}[matrice]
	Una matrice $ m \times n $ a coefficienti in $ \K $ è un tabella ordinata di $ m $ righe e $ n $ colonne i cui elementi appartengono ad un campo $ \K $. L'insieme delle matrici $ m \times n $ a coefficienti nel campo $ \K $ viene indicato con $ \mathrm{Mat}_{m \times n}{(K)} $ ed è uno spazio vettoriale. \\ Dati $ a_{ij} \in \K $ con $ i = 1, \ldots, m $ e $ j = 1, \ldots, n $ diremo che $ A \in \mathrm{Mat}_{m \times n} (\K) $ e scriveremo
	\[ A = (a_{ij}) =
	\begin{pmatrix}
		a_{11} & \cdots  & a_{1n} \\
		\vdots & \ddots & \vdots \\
		a_{m1} & \cdots  & a_{mn} \\
	\end{pmatrix}\]
\end{definition}

\begin{definition}[matrice diagonale e identità]
	contenuto...
\end{definition}

\begin{definition}[prodotto tra matrici]
	contenuto...
\end{definition}

\begin{propriety}
	\textsf{proprietà del prodotto ($ n \times n $ stabile rispetto al prodotto)}
\end{propriety}

\begin{definition}[matrice trasposta]
	contenuto...
\end{definition}

\begin{propriety}[della trasposta]
	contenuto...
\end{propriety}

\begin{definition}[matrici coniugate]
	Due matrici $ A $ e $ B $ si dicono coinugate se esiste una matrice $ P $ invertibile tale che \[B = P^{-1} A P.\] Matrici coniugate rappresentano la stessa applicazioni lineari viste in due basi diverse. 
\end{definition}

\begin{definition}[traccia]
	Sia $ M $ una matrice quadrata $ n \times n $. La traccia di $ M $ è la somma degli elementi sulla diagonale 
	\[\tr{(M)} = \tr{ 
	\begin{pmatrix}
	a_{11} & \cdots  & a_{1n} \\
	\vdots & \ddots & \vdots \\
	a_{n1} & \cdots  & a_{nn} \\
	\end{pmatrix}}
	= a_{11} + \ldots + a_{nn}\]
\end{definition}

\begin{propriety}[della traccia]
	La traccia gode delle seguenti proprietà
	\begin{enumerate}[label=(\roman*)]
		\item $ \tr (A + B) = \tr(A) + \tr(B) $ e $ \tr(\lambda A) = \lambda \, \tr (A) $
		\item $ \tr(\prescript{t}{}{A}) = \tr (A) $
		\item $ \tr (AB) = \tr (BA) $
	\end{enumerate}
\end{propriety}

\begin{thm}[invarianza della traccia per coniugio]
	Se $ A $ e $ B $ sono matrici coniugate, allora $ \tr (A) = \tr (B) $
\end{thm}

Roba su riduzione a scalini...

\begin{thm} \textbf{*}
	Tutte e sole le matrici $ A \in \mathrm{Mat}_{n \times n}(\K) $ che commutano con ogni matrice $ B \in  \mathrm{Mat}_{n \times n}(\K) $ sono multipli dell'identità. \[AB = BA, \ \forall B \in \mathrm{Mat}_{n \times n}(\K) \quad \iff \quad \exists \lambda \in \K : A = \lambda \Id\]
\end{thm}