% SISTEMI LINEARI

\begin{definition}[Sistema lineare omogeneo]
	contenuto...
\end{definition}

\begin{definition}[Matrice associata al sistema lineare omogeneo]
	contenuto...
\end{definition}

\begin{prop}[Dimensione delle soluzioni]
	Sia $ M $ la matrice associata ad un sistema lineare omogeneo con $ n $ incognite. Indicando con $ S $ lo spazio delle soluzioni del sistema lineare vale \[\dim{S} = n - \rg{M}\]
\end{prop}

%	SUPERFLUO...
%\begin{definition}[sottospazio ortogonale]
%	contenuto... + \textsf{il sottospazio ortogonale è l'inieme delle soluzioni del sistema omogeneo}
%\end{definition}

\begin{definition}[Sistema lineare non omogeneo]
	contenuto...
\end{definition}

\begin{definition}[Matrice completa e incompleta associata]
	contenuto...
\end{definition}

\begin{thm}[Rouchè-Capelli]
	Un sistema omogeneo ammette soluzione se e solo se il rango della matrice completa e incompleta coincidono.
\end{thm}

\begin{thm}[Insieme soluzioni del sistema non omogeneo]
	Sia $ S $ l'insieme delle soluzioni del sistema non omogeneo e $ S_0 $ l'insieme delle soluzioni del sistema omogeneo associato. Supposto $ S \neq \emptyset $, preso un qualunque $ v \in S $ vale \[S = v + S_0\]
\end{thm}

\begin{definition}[Sottospazio affine]
	Sia $ V $ un $ \K $-spazio vettoriale, $ U $ un suo sottospazio e $ v \in V - U $ ($ v \neq 0 $) si dice che l'insieme $ v + U $ è un sottospazio affine di $ V $. Per convenzione si pone $ \dim{(v + U)} = \dim{U} $.
\end{definition}