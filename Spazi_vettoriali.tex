%%% SPAZI VETTORIALI

\begin{definition}[Campo]
	Un campo è una terna $ (\K, +, \cdot) $ dove $ \K $ è un insieme su cui sono definite due operazioni di \emph{somma} $ + \colon \K \times \K \to \K $ e \emph{prodotto} $ \cdot \colon \K \times \K \to \K $ che associano a due elementi dell'insieme un altro elemento dell'insieme, ovvero tali che
	\begin{enumerate}
		\item $ \forall x, y \in \K \Rightarrow x + y \in \K $
		\item $ \forall x, y \in \K \Rightarrow x \cdot y \in \K $
	\end{enumerate}
	e che rispettano le seguenti proprietà
	\begin{enumerate}[label = (\roman*)]
		\item \emph{associativa}
		\item \emph{commutativa}
		\item \emph{esistenza degli elementi neutri}
		\item \emph{opposto}
		\item \emph{inverso}
		\item \emph{distributiva del prodotto rispetto alla sommma}
	\end{enumerate}
\end{definition}

\begin{definition}[Spazio vettoriale]
	Uno spazio vettoriale $ V $ su un campo $ \K $ o $ \K $-spazio vettoriale è una quaterna $ (V, \K, +, \cdot) $ dove $ K $ è un campo e $ V $ è un insieme non vuoto su cui sono definite due operazioni di 
	\begin{enumerate}
		\item \emph{somma} $ + \colon V \times V \to V $ tra elementi di $ V $ tale che $ \forall v, w \in V \Rightarrow v + w \in V $
		\item \emph{prodotto per scalare} $ \cdot \colon \K \times V \to V $ tale che $ \forall v \in V, \forall \lambda \in \K \Rightarrow \lambda \cdot v = \lambda v \in V $
	\end{enumerate}
	che devono rispettare le seguenti proprietà:
	\begin{enumerate}[label=(\roman*)]
		\item \emph{associativa della somma}
		\item \emph{commutativa della somma}
		\item \emph{esistenza dell'elemento neutro della somma}
		\item \emph{opposto della somma}
		\item \emph{distributiva del prodotto rispetto alla somma}
		\item \emph{associativa del prodotto}
		\item \emph{elemento neutro del prodotto}
	\end{enumerate}
	Chiameremo \emph{vettori} gli elementi di $ V $. \\
	\textsf{Nota: dove non specificato intenderemo sempre che $ V $ è uno spazio vettoriale su un campo $ \K $.}
\end{definition}

\begin{propriety} Uno spazio vettoriale gode delle seguenti proprietà:
	\begin{enumerate}
		\item Unicità dell'elemento neutro
		\item Unicità dell'opposto
		\item $ 0_{\K} \cdot v = O_V $
		\item $ (-1_{\K}) \cdot v = -v $
	\end{enumerate}
\end{propriety}


% SPOSTATO 
%\begin{definition}[Sottospazio vettoriale]
%	Sia $ V $ un $ \K $-spazio vettoriale. Diciamo che $ W \subseteq V $ è un sottospazio vettoriale di $ V $ se valgono le seguenti proprietà
%	\begin{enumerate}[label=(\roman*)]
%		\item $ \forall v, w \in W \Rightarrow v + w \in W $
%		\item $ \forall v \in W, \forall \lambda \in \K \Rightarrow \lambda v \in W $
%		\item $ O_V \in W $
%	\end{enumerate}
%\end{definition}


%\begin{thm}[intersezione di sottospazi]
%	\textsf{Intersezione di sottospazi è un sottospazio vettoriale}
%\end{thm}
%
%\begin{thm}[somma di sottospazi]
%	\textsf{Somma di sottospazi è un sottospazio vettoriale}
%\end{thm}

\begin{definition}[Combinazione lineare]
	Sia $ V $ un $ \K $-spazio vettoriale, siano $ v_1, \ldots , v_n \in \nobreak V $ e $ \lambda_1, \ldots , \lambda_n \in \K $. 
	Si dice combinazione lineare dei $ v_i $ un vettore $ v \in V $ tale che 
	\[v = \sum_{i=1}^{n} \lambda_i v_i = \lambda_1 v_1 + \ldots + \lambda_n v_n \]
\end{definition}


\begin{definition}[Span]
	\[Span\{v_1, \ldots , v_n\} = \left\{ v \in V: \exists \lambda_1, \ldots , \lambda_n \in \K: v = \sum_{i=1}^{n} \lambda_i v_i \right\}\]
\end{definition}

\begin{prop}[Unicità della combinazione lineare]
	Sia $ V $ un $ \K $-spazio vettoriale, siano $ v_1, \ldots , v_n \in \nobreak V $. Sia 
	$ w \in Span\{v_1, \ldots , v_n\}$ allora $\exists !  \lambda_1, \ldots , \lambda_n \in \K$ tali che
	\[w = \sum_{i=1}^{n} \lambda_i v_i = \lambda_1 v_1 + \ldots + \lambda_n v_n. \]  
\end{prop}

\begin{thm}[Proprietà dello span]
	Sia $ V $ un $ \K $-spazio vettoriale e siano $ v_1, \ldots , v_n \in \nolinebreak V $. 
	Allora $ Span\{v_1, \ldots , v_n\} $ è il più piccolo sottospazio vettoriale di $ V $ 
	che contiene tutti i $ v_i $.
\end{thm}

\begin{definition}[Dipendenza e indpendenza lineare] 
	Sia $ V $ un $ \K $-spazio vettoriale e siano $ v_1, \ldots , v_n \in V $. Si dice che $ v_1, \ldots, v_n $ 
	sono linearmente dipendenti se $ \exists \lambda_1, \ldots , \lambda_n \in \K $ non tutti nulli tali che 
	\[\lambda_1 v_1 + \ldots + \lambda_n v_n = O_V.\] Analogamente si dice che $ v_1, \ldots, v_n $ sono linearmente 
	indipendenti se \[\lambda_1 v_1 + \ldots + \lambda_n v_n = O_V \iff \lambda_1 = \ldots = \lambda_n = 0.\]	
\end{definition}

\begin{definition}[Base]
	Sia $ V $ un $ \K $-spazio vettoriale. Un insieme $ \{v_1, \ldots, v_n\} $ si dice base di $ V $ se
	\begin{enumerate}[label=(\roman*)]
		\item $ v_1, \ldots, v_n $ sono linearmente indipendenti
		\item $ Span\{v_1, \ldots , v_n\} = V $
	\end{enumerate}
\end{definition}

% Superfluo...
%\begin{definition}[sottoinsieme massimale]
%	\textsf{se aggiungo un vettore l'insieme non è più linearmente indipendente}	
%\end{definition}

\begin{lemma}[Base è indipendente massimale]
	Sia $A \subset V$. $A$ è indipendente massimale se e solo se è una base. 
\end{lemma}

\begin{lemma}[Base è generatore minimale]
        Sia $A \subset V$. $A$ è generatore minimale se e solo se è una base.
\end{lemma}

\begin{thm}
	Ogni\footnote{Dimostrato solo per spazi finitamente generati, ma vero in generale. In seguito ci si riferirà solemante a spazi finitamente generati.} 
	spazio vettoriale ammette una base.
\end{thm}

\begin{lemma}[Algoritmo di scambio]
	Sia $ V $ un $ \K $-spazio vettoriale e siano $ A = \{v_1, \ldots, v_m\} $ e $ B = \{  w_1, \ldots, w_n \} $ due insiemi linermente indipendenti,
	tali che $ B \subseteq Span(A) $. Allora $\exists B' \subseteq A $, tale che
	\[ A' = \left ( A \\ B' \right ) \cup B \] è linearmente indipendente e $Span(A') = Span(A)$. 
\end{lemma}

\begin{corollary}
        Sia $ V $ un $ \K $-spazio vettoriale e siano $ A = \{v_1, \ldots, v_m\} $ e $ B = \{  w_1, \ldots, w_n \} $ due insiemi linermente indipendenti,
        tali che $ B \subseteq Span(A) $. Allora $ \#B \le \#A $.
\end{corollary}

\begin{thm}[Dimensione di una base]
        Sia $ V $ un $ \K $-spazio vettoriale. Supponiamo di avere due basi di $ V $, una con $ n $ elementi e una con $ m $ elementi. Allora $ n = m $.
\end{thm}

\begin{definition}[Dimensione]
	Sia $ V $ un $ \K $-spazio vettoriale avente una base costituita da $ n $ vettori. Allora diremo che $ V $ ha dimensione $ n $ e 
	scriveremo $ \dim{V} = n $.
\end{definition}

\begin{thm}[Estrazione di una base] 
	Sia $ V $ un $ \K $-spazio vettoriale di dimensione $ n $. Sia $A$ un insieme di generatori di $V$, allora $\exists B \subseteq A$ tale che $B$ è una base.
\end{thm}

\begin{thm}[Completamento ad una base]
        Sia $ V $ un $ \K $-spazio vettoriale di dimensione $ n \geq 2 $. Sia $ r $ un intero positivo con $ 0 < r < n $. 
        Dati $ r $ vettori $ v_1, \ldots , v_r \in V $ linearmente indipendenti è possibile completarli ad una base di $ V $, 
        ossia trovare vettori $ v_{r+1}, \ldots, v_n $ tali che $ \{v_1, \ldots , v_r, v_{r+1}, \ldots , v_n\} $ è base di $ V $.
\end{thm}

\begin{prop}
	Sia $ V $ un $ \K $-spazio vettoriale di dimensione $ n $. Allora valgono i seguenti fatti:
	\begin{enumerate}[label=(\roman*)]
                \item Se $ v_1, \ldots, v_m $ sono linearmente indipendenti allora $m \le n$;
                \item Ogni insieme con $>n$ elementi è linearmente dipendente;
                \item Se $ v_1, \ldots, v_m $ generano allora $m \ge n$;
		\item Ogni insieme con $<n$ elementi non genera;
		\item Ogni insieme di $n$ elementi che genera è una base.
        \end{enumerate}
\end{prop}

