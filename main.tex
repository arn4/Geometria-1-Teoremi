\documentclass[9pt, a4paper]{article}
\usepackage[a4paper,top=3.cm,bottom=3.cm,left=3cm,right=3cm]{geometry}

\usepackage{mathstyle}

% THEOREM ENVIROMMENT	
\newtheoremstyle{mythm}{\topsep}{\topsep}{\rmfamily}{}{\bfseries}{}{.5em}{}

\theoremstyle{mythm}
\newtheorem*{axiom}{Assioma}
\newtheorem*{definition}{Definizione}
\newtheorem*{propriety}{Proprietà}
\newtheorem*{thm}{Teorema}
\newtheorem*{corollary}{Corollario}
\newtheorem*{prop}{Proposizione}
\newtheorem*{lemma}{Lemma}
\newtheorem*{fatto}{Fatto \textbf{*}}

\newenvironment{skp}{\paragraph{Sketch dimostrazione:}}{}

\title{\textsc{Teoremi Geometria 1}}
\author{Alessandro Piazza \thanks{alessandro.piazza@sns.it} \\ 
		Luca Arnaboldi \thanks{luca.arnaboldi@sns.it}}


\begin{document}

\maketitle

\begin{abstract}
	Raccolta di Teoremi e altri risultati raccolti dal corso di Geometria 1.
	
	Il documento è un riadattamento del documento prodotto da Alessandro Piazza per il Corso B, dove sono state aggiunte alcune modifiche per renderlo più adatto al Corso A. 
	Sono stati aggiunti anche degli sketch di dimostrazione per i Fatti e i Teoremi più complicati. 
	\begin{center}
		\textsc{Attenzione: questa è una bozza, ci sono molti errori}
	\end{center}
\end{abstract}

\tableofcontents

\clearpage

\section{Spazi Vettoriali}
% SAPZI VETTORIALI

\begin{definition}[campo] \footnote{Riassunto della sezione 1}
	Un campo è una terna $ (\K, +, \cdot) $ dove $ \K $ è un insieme su cui sono definite due operazioni di \emph{somma} $ + \colon \K \times \K \to \K $ e \emph{prodotto} $ \cdot \colon \K \times \K \to \K $ che associano a due elementi dell'insieme un altro elemento dell'insieme, ovvero tali che
	\begin{enumerate}
		\item $ \forall x, y \in \K \Rightarrow x + y \in \K $
		\item $ \forall x, y \in \K \Rightarrow x \cdot y \in \K $
	\end{enumerate}
	e che rispettano le seguenti proprietà
	\begin{enumerate}[label = (\roman*)]
		\item \emph{associativa}: $ \forall x, y, z \in \K, \ x + (y + z) = (x + y) + z \text{ e } x \cdot (y \cdot z) = (x \cdot y) \cdot z $;
		\item \emph{commutativa}: $ \forall x, y \in \K, \ x + y = y + x \text{ e } x \cdot y = y \cdot x $;
		\item \emph{esistenza degli elementi neutri}: $ \exists 0_{\K}, 1_{\K} \in \K: 0_{\K} \neq 1_{\K} $ tali che $ \forall x \in \K, \ x + 0_{\K} = 0_{\K} + x = x $ e $ \forall x \in \K, \ x \cdot 1_{\K} = 1_{\K} \cdot x = x $ (tali elementi sono unici e per semplicità vengono indicati con $ 0 $ e $ 1 $);
		\item \emph{opposto}: $ \forall x \in \K, \exists y \in \K : x + y = 0_{\K} $ che viene indicato con $ - x $;
		\item \emph{inverso}: $ \forall x \in \K : x \neq 0_{\K}, \exists y \in \K : x \cdot y = 1_{\K} $ che viene indicato con $ \frac{1}{x} $ o $ x^{-1} $;
		\item \emph{distributiva del prodotto rispetto alla somma}: $ \forall x, y, z \in \K, \ x \cdot (y + z) = x \cdot y + x \cdot z $.
	\end{enumerate}
\end{definition}

\begin{definition}[spazio vettoriale]
	Uno spazio vettoriale $ V $ su un campo $ \K $ o $ \K $-spazio vettoriale è una quaterna $ (V, \K, +, \cdot) $ dove $ \K $ è un campo e $ V $ è un insieme non vuoto su cui sono definite due operazioni di 
	\begin{enumerate}
		\item \emph{somma} $ + \colon V \times V \to V $ tra elementi di $ V $ tale che $ \forall v, w \in V \Rightarrow + (v, w) = v + w \in V $
		\item \emph{prodotto per scalare} $ \cdot \colon \K \times V \to V $ tale che $ \forall v \in V, \forall \lambda \in \K \Rightarrow \cdot (\lambda, v) = \lambda \cdot v = \lambda v \in V $
	\end{enumerate}
	che devono rispettare le seguenti proprietà:
	\begin{enumerate}[label=(\roman*)]
		\item \emph{associativa della somma}: $ \forall v, w, u \in V, \ v + (w + u) = (v + w) + u $;
		\item \emph{commutativa della somma}: $ \forall v, w \in V, \ v + w = w + v $;
		\item \emph{esistenza dell'elemento neutro della somma}: $ \exists O_{V} \in V : \forall v \in V, \ v + O_{V} = O_{V} + v = v $;
		\item \emph{opposto della somma}: $ \forall v \in V, \exists w \in W : v + w = O_{V} $ che viene indicato con $ w = -v $;
		\item \emph{distributiva del prodotto rispetto alla somma}: $ \forall \lambda \in \K, \forall v, w \in V, \ \lambda (v + w) = \lambda v + \lambda w $;
		\item \emph{distributiva del somma sul campo rispetto al prodotto}: $ \forall \lambda, \mu \in \K, \forall v \in V, \ (\lambda + \mu) v = \lambda v + \mu v $;
		\item \emph{associativa del prodotto}: $ \forall \lambda, \mu \in \K, \forall v \in V, \ (\lambda \mu) v = \lambda (\mu v) $;
		\item \emph{elemento neutro del prodotto}: $ \exists 1_{\K} \in \K : \forall v \in V, \ 1_{\K} \cdot v = v $;
	\end{enumerate}
	Chiameremo \emph{vettori} gli elementi di $ V $. \\
	\textsf{Nota: dove non specificato intenderemo sempre che $ V $ è uno spazio vettoriale su un campo $ \K $.}
\end{definition}

\begin{propriety} Uno spazio vettoriale gode delle seguenti proprietà:
	\begin{enumerate}
		\item Unicità dell'elemento neutro
		\item Unicità dell'opposto
		\item $ 0_{\K} \cdot v = O_V $
		\item $ (-1_{\K}) \cdot v = -v $
	\end{enumerate}
\end{propriety}

\begin{definition}[sottospazio vettoriale]
	Sia $ V $ un $ \K $-spazio vettoriale. Diciamo che $ W \subseteq V $ è un sottospazio vettoriale di $ V $ se valgono le seguenti proprietà
	\begin{enumerate}[label=(\roman*)]
		\item $ \forall v, w \in W \Rightarrow v + w \in W $
		\item $ \forall v \in W, \forall \lambda \in \K \Rightarrow \lambda v \in W $
		\item $ O_V \in W $
	\end{enumerate}
\end{definition}

\begin{thm}[intersezione di sottospazi]
	Sia $ V $ un $ \K $-spazio vettoriale e $ W, U \subseteq V $ sottospazi di $ V $. Allora $ W \cap U $ è sottospazio di $ V $. 
\end{thm}

\begin{thm}[somma di sottospazi]
	Sia $ V $ un $ \K $-spazio vettoriale e $ W, U \subseteq V $ sottospazi di $ V $. Allora \[W + U = \{v \in V : \exists w \in W, u \in U : v = w + u\}\] è un sottospazio vettoriale di $ V $. 
\end{thm}

\begin{definition}[combinazione lineare]
	Sia $ V $ un $ \K $-spazio vettoriale, siano $ v_1, \ldots , v_n \in V $ e $ \lambda_1, \ldots , \lambda_n \in \K $. Si dice combinazione lineare dei $ v_i $ un vettore $ v \in V $ tale che \[v = \sum_{i = 1}^{n} \lambda_i v_i = \lambda_1 v_1 + \ldots + \lambda_k v_n.\]
\end{definition}

\begin{definition}[Span]
	Sia $ V $ un $ \K $-spazio vettoriale e siano $ v_1, \ldots, v_k \in V $. Si dice $ Span{\{v_1, \ldots , v_n\}} $ l'insieme di tutte le possibili combinazioni lineari dei $ v_i $. Formalmente
	\[Span{\{v_1, \ldots , v_n\}} = \left\{ v \in V: \exists \lambda_1, \ldots , \lambda_n \in \K: v = \sum_{i=1}^{n} \lambda_i v_i \right\}.\]
\end{definition}

\begin{thm}[proprietà dello span]
	Sia $ V $ un $ \K $-spazio vettoriale e siano $ v_1, \ldots , v_n \in \nolinebreak V $. Allora $ Span\{v_1, \ldots , v_n\} $ è il più piccolo sottospazio vettoriale di $ V $ che contiene tutti i $ v_i $.
\end{thm}

\begin{definition}[dipendenza e indipendenza lineare] 
	Sia $ V $ un $ \K $-spazio vettoriale e siano $ v_1, \ldots , v_n \in V $. Si dice che $ v_1, \ldots, v_n $ sono linearmente dipendenti se $ \exists \lambda_1, \ldots , \lambda_n \in \K $ non tutti nulli tali che \[\lambda_1 v_1 + \ldots + \lambda_n v_n = O_V.\] Analogamente si dice che $ v_1, \ldots, v_n $ sono linearmente indipendenti se \[\lambda_1 v_1 + \ldots + \lambda_n v_n = O_V \Rightarrow \lambda_1 = \ldots = \lambda_n = 0.\]	
\end{definition}

\begin{definition}[base]
	Sia $ V $ un $ \K $-spazio vettoriale. Un insieme $ \{v_1, \ldots, v_n\} $ si dice base di $ V $ se
	\begin{enumerate}[label=(\roman*)]
		\item $ v_1, \ldots, v_n $ sono linearmente indipendenti;
		\item $ Span\{v_1, \ldots , v_n\} = V $ (generano).
	\end{enumerate}
\end{definition}

\begin{thm}[unicità della combinazione lineare]
	Sia $ V $ un $ \K $-spazio vettoriale e siano $ v_1, \ldots , v_n \in V $ linearmente indipendenti. Se $ \exists \lambda_1, \ldots, \lambda_n \in \K $ e $ \exists \mu_1, \ldots, \mu_n \in \K $ tali che $ \sum_{i = 1}^{n} \lambda_i v_i = \sum_{i = 1}^{n} \mu_i v_i $, allora $ \lambda_i = \mu_i $ per ogni $ i = 1, \ldots, n $. Dunque la scrittura di un vettore come comb9inazione lineare di vettori linearmente indipendenti è unica.
\end{thm}

\begin{definition}[sottoinsieme massimale]
	Sia $ V $ un $ \K $-spazio vettoriale e siano $ v_1, \ldots , v_n \in V $. Diciamo che l'insieme $ \{v_1, \ldots v_r\} $ con $ r \in \N $ e $ r \leq n $ è un sottoinsieme massimale di vettori linearmente indipendenti se $ v_1, \ldots, v_r $ sono linearmente indipendenti e se $ \forall i \in \N : r < i \leq n $, $ v_1, \ldots, v_r, v_i $ sono linearmente dipendenti. 
\end{definition}

\begin{thm}
	Sia $ V $ un $ \K $-spazio vettoriale e siano $ v_1, \ldots , v_m \in V : Span{\{v_1, \ldots, v_m\}} = V $ (generano). Sia $ \{v_1, \ldots, v_n\} $ un sottoinsieme di $ \{v_1, \ldots, v_m\} $ linearmente indipendente e massimale. Allora $ \{v_1, \ldots, v_n\} $ è una base di $ V $. 
\end{thm}

\begin{thm}
	Sia $ V $ un $ \K $-spazio vettoriale e sia $ \{v_1, \ldots, v_n\} $ una base di $ V $. Se $ w_1, \ldots, w_m $, con $ m > n $, sono vettori di $ V $, allora $ w_1, \ldots, w_m $ sono linearmente dipendenti.
\end{thm}
\begin{proof}
	Se $ \exists j \in \{1, \ldots, m\} $ tale che $ w_j = 0 $ allora $ \forall \lambda \in \K $ si ha $ 0 w_1 + \ldots + \lambda w_j + \ldots + 0 w_m = O_V $ e quindi l'enunciato risulta verificato. Supponiamo quindi che $ w_1, \ldots, w_m $ siano tutti non nulli. Per assurdo $ w_1, \ldots, w_m $ sono linearmente indipendenti. Poiché $ \{v_1, \ldots, v_n\} $ è una base di $ V $ allora $ w_1 = a_1 v_1 + \ldots a_n v_n $ con $ a_1, \ldots, a_n \in \K $ non tutti nulli. Supponiamo quindi \emph{wlog} $ a_1 \neq 0 $, allora \[v_1 = \frac{1}{a_1} w_1 - \frac{a_2}{a_1} v_2 - \ldots \frac{a_n}{a_1}v_n.\] Allora $ v_1 \in Span{\{w_1, v_2, \ldots, v_n\}} $ così $ V = Span{\{v_1, \ldots, v_n\}} \subseteq Span{\{w_1, v_2, \ldots, v_n\}} $ e quindi $ Span{\{w_1, v_2, \ldots, v_n\}} = V $. \\
	L'idea è di rimpiazzare tutti i $ v_1, \ldots, v_n $ con i $ w_1, \ldots, w_n $ così che $ w_1, \ldots, w_n $ generino $ V $. Procediamo in per induzione: supponiamo esista $ r \in \N : r \leq n $ tale che $ w_1, \ldots, w_r, v_{r + 1}, \ldots, v_n $ generino $ V $. Allora esistono $ b_1, \ldots, b_n \in \K $ tali che $ w_{r + 1} = b_1 w_1 + \ldots b_{r} w_r + b_{r + 1} v_{r + 1} + \ldots b_n v_n $. Osserviamo che almeno uno tra $ b_{r + 1}, \ldots, b_n $ è non nullo (se fossero tutti nulli otterremmo una relazione di lineare dipendenza tra $ w_1, \ldots, w_m $). Così \[v_{r + 1} = - \frac{b_1}{b_{r + 1}} w_1 - \ldots - \frac{b_r}{b_r + 1} + \frac{1}{b_{r + 1}} w_{r + 1} - \frac{b_{r + 2}}{b_{r + 1}} v_{r + 2} - \ldots \frac{b_n}{b_{r + 1}} v_{n}.\] Dunque $ v_{r + 1} \in Span{\{w_1, \ldots, w_{r + 1}, v_{r + 2}, \ldots, v_n\}} $ così $ V = Span{\{w_1, \ldots, w_r, v_{r + 1}, \ldots, v_n\} }\subseteq Span{\{w_1, \ldots, w_{r + 1}, v_{r + 2}, \ldots, v_n\}} $ quindi $ w_1, \ldots, w_{r + 1}, v_{r + 2}, \ldots, v_n $ generano $ V $. \\
	Per induzione su $ r $ allora $ Span{\{w_1, \ldots, w_n\}} = V $. Ma allora per $ m > n $ esistono $ d_1, \ldots, d_n \in \K $ non tutti nulli tali che $ w_m = d_1 w_1 + \ldots + d_n w_n $. Così $ w_1, \ldots, w_m, w_n $ non sono linearmente indipendenti da cui l'assurdo. 
\end{proof}


\begin{corollary}
	Sia $ V $ un $ \K $-spazio vettoriale. Supponiamo di avere due basi di $ V $, una con $ n $ elementi e una con $ m $ elementi. Allora $ n = m $.
\end{corollary}

\begin{definition}[dimensione]
	Sia $ V $ un $ \K $-spazio vettoriale avente una base costituita da $ n $ vettori. Allora diremo che $ V $ ha dimensione $ n $ e scriveremo $ \dim{V} = n $. \\ 
	\textsf{Nota: dove non specificato lo spazio considerato ha dimensione finita $ n $.}
\end{definition}

\begin{thm}
	Sia $ V $ un $ \K $-spazio vettoriale di $ \dim{V} = n $. Se $ v_1, \ldots, v_n $ generano $ V $ allora $ \{v_1, \ldots, v_n\} $ è una base di $ V $.
\end{thm}

\begin{thm}
	Sia $ V $ un $ \K $-spazio vettoriale. Se $ \{v_1, \ldots, v_n\} $ sono un insieme massimale di vettori di $ V $ linearmente indipendenti allora $ \{v_1, \ldots, v_n\} $ è una base di $ V $. 
\end{thm}

\begin{thm}
	Sia $ V $ un $ \K $-spazio vettoriale di $ \dim{V} = n $. Se $ v_1, \ldots, v_n $ un insieme di vettori di $ V $ linearmente indipendenti allora $ \{v_1, \ldots, v_n\} $ è una base di $ V $.
\end{thm}

\begin{thm}[completamento ad una base]
	Sia $ V $ un $ \K $-spazio vettoriale di $ \dim{V} = n $. Sia $ r $ un intero positivo con $ 0 < r < n $. Dati $ r $ vettori $ v_1, \ldots , v_r \in V $ linearmente indipendenti è possibile completarli ad una base di $ V $, ossia trovare vettori $ v_{r+1}, \ldots, v_n $ tali che $ \{v_1, \ldots , v_r, v_{r+1}, \ldots , v_n\} $ è base di $ V $.
\end{thm}

\clearpage

\section{Sottospazi}
\begin{definition}[Sottospazio vettoriale]
       Sia $ V $ un $ \K $-spazio vettoriale. Diciamo che $ W \subseteq V $ è un sottospazio vettoriale di $ V $ se valgono le seguenti proprietà
       \begin{enumerate}[label=(\roman*)]
               \item $ \forall v, w \in W \Rightarrow v + w \in W $
               \item $ \forall v \in W, \forall \lambda \in \K \Rightarrow \lambda v \in W $
               \item $ O_V \in W $
       \end{enumerate}
\end{definition}

\begin{definition}[Somma tra sottospazi]
	Sia V uno spazio vettoriale, e siano $U$ e $W$ due sottospazi. Allora si definisce somma
	\[ U+W = \{ u + w | u \in U, w \in W \}. \]
\end{definition}

\begin{prop}[Operazioni tra sottospazi]
	Valgono i seguenti fatti: 
	\begin{enumerate}[label=(\roman*)]
		\item Intersezione di sottospazi è un sottospazio vettoriale;
		\item Somma di sottospazi è un sottospazio vettoriale;
		\item Unione di sottospazi è un sottospazio vettoriale se e solo se uno è contemnuto nell'altro.
	\end{enumerate}
\end{prop}


\begin{thm}[Formula di Grassmann]
	Siano $ A $ e $ B $ sottospazi vettoriali di $ V $ su un campo $ \K $. Vale \[\dim{A} + \dim{B} = \dim{(A + B)} + \dim{(A \cap B)}\]
\end{thm}

\begin{definition}[Somma diretta]
	Dati $ A $ e $ B $ sottospazi di $ V $ su un campo $ \K $, si dice che $ A $ e $ B $ sono in somma diretta, e si scriverà $ A \oplus B$, se $ A \cap B = \{O_V\} $.
	In modo del tutto equivalente $ A $ e $ B $ sono in somma diretta se e solo se $ \dim{A} + \dim{B} =  \dim{(A + B)} $.	
\end{definition}

\begin{prop}[Unicità della decomposizione]
	Sia $ Z = U \oplus V $ e $ z \in Z$. Allora $ \exists! u \in U, w \in W$ tali che $ z = u + v $.
\end{prop}
	
% SUPERFLUO
%\begin{definition}[Somma diretta di $ k $ sottospazi]
%	$ U_1, \ldots , U_k $ sottospazi di $ V $ su un campo $ \K $ si dicono essere insomma diretta se $ \forall i \in \{1, \ldots, k\} $ vale \[U_i \cap (U_1 + \ldots + \hat{U}_i + \ldots + U_k) = \{O_V\}\]
%	In modo equivalente $ U_1, \ldots , U_k $ sono insomma diretta se e solo se \[\dim{U_1} + \ldots + \dim{U_k} = \dim{(U_1 + \ldots + U_k)}\]
%\end{definition}

\begin{definition}[Complementare di un sottospazio]
	Sia $ A $ un sottospazio di $ V $ su un campo $ \K $. Un complementare di $ A $ è un sottospazio $ B $ di $ V $ tale che
	\begin{enumerate}[label=(\roman*)]
		\item $ A \cap B = \{O_V\} $ ($ A $ e $ B $ sono in somma diretta)
		\item $ A + B = V $
	\end{enumerate}
	In tal caso scriveremo che $ A \oplus B = V $. 
\end{definition}


\clearpage

\section{Matrici}
% MATRICI

\begin{definition}[Matrice]
	Una matrice $ m \times n $ a coefficienti in $ \K $ è un tabella ordinata di $ m $ righe e $ n $ colonne i cui elementi appartengono ad un campo $ \K $. L'insieme delle matrici $ m \times n $ a coefficienti nel campo $ \K $ viene indicato con $ \mathrm{Mat}_{m \times n}{(K)} $ ed è uno spazio vettoriale. \\ Dati $ a_{ij} \in \K $ con $ i = 1, \ldots, m $ e $ j = 1, \ldots, n $ diremo che $ A \in \mathrm{Mat}_{m \times n} (\K) $ e scriveremo
	\[ A = (a_{ij}) =
	\begin{pmatrix}
		a_{11} & \cdots  & a_{1n} \\
		\vdots & \ddots & \vdots \\
		a_{m1} & \cdots  & a_{mn} \\
	\end{pmatrix}\]
\end{definition}

\begin{definition}[Matrice diagonale e Identità]
	contenuto...
\end{definition}

\begin{definition}[Matrice trasposta]
	contenuto...
\end{definition}

\begin{propriety}[della trasposta]
	contenuto...
\end{propriety}

\begin{definition}[Prodotto tra matrici]
	contenuto...
\end{definition}

\begin{propriety}[del prodotto tra matrici]
	Il prodotto tra matrici gode delle seguenti proprietà:
	\begin{enumerate}[label=(\roman*)]
		\item $ A(B+C) = AB + AC $;
		\item $ (\alpha A)B = A(\alpha B) = \alpha AB $;
		\item $ (AB)C = A(BC) $;
		\item $ ^t(AB) = ^tB^tA $; %TODO: sistemare i trasposti
		\item $(AB)^{-1} = B^{-1}A^{-1}.$
	\end{enumerate}
\end{propriety}

\begin{definition}[Matrice inversa]
	Si chiama inversa di una matrice quadrata $A$ e si indica con $A^{-1}$ la matrice tale che
	\[AA^{-1}=A^{-1}A=Id.\]	
\end{definition}

% SPOSTATO
%\begin{definition}[Matrici coniugate]
%	Due matrici $ A $ e $ B $ si dicono coinugate se esiste una matrice $ P $ invertibile tale che %	\[B = P^{-1} A P.\] Matrici coniugate rappresentano la stessa applicazioni lineari viste in due basi diverse. 
%\end{definition}

\begin{definition}[Traccia]
	Sia $ M $ una matrice quadrata $ n \times n $. La traccia di $ M $ è la somma degli elementi sulla diagonale 
	\[\tr{(M)} = \tr{ 
	\begin{pmatrix}
	a_{11} & \cdots  & a_{1n} \\
	\vdots & \ddots & \vdots \\
	a_{n1} & \cdots  & a_{nn} \\
	\end{pmatrix}}
	= a_{11} + \ldots + a_{nn}\]
\end{definition}

\begin{propriety}[della traccia]
	La traccia gode delle seguenti proprietà
	\begin{enumerate}[label=(\roman*)]
		\item $ \tr (A + B) = \tr(A) + \tr(B) $ e $ \tr(\lambda A) = \lambda \, \tr (A) $
		\item $ \tr(\prescript{t}{}{A}) = \tr (A) $
		\item $ \tr (AB) = \tr (BA) $. Più in generale una permutazione ciclica del prodotto non cambia la traccia.
		\item la traccia è inariante per coniugio
			\footnote{Viene definito in seguito il coniugio};
	\end{enumerate}
\end{propriety}

\begin{definition}[Rango]
	Si dice rango di una matrice la dimensione dello Span dei vettori colonna.
\end{definition}

\begin{definition}[Riduzione per righe]
	contenuto...
\end{definition}

\begin{definition}[Riduzione per colonne]
	contenuto...
\end{definition}

\begin{definition}[Pivot]
	contenuto...
\end{definition}

\begin{thm}
	Sia $A$ una matrice, vale la seguente uguaglianza:
	\[ \rg(A) = \#\text{pivot} = \#\text{(righe $\ne 0$)} = \dim \left ( Span (\text{vettori riga} ) \right ) \]
\end{thm}

\begin{corollary}
	Per ogni matrice $A$ si ha che \[ \rg(A) = \rg(^t A).\]
\end{corollary}

\begin{prop}
	Le righe non nulle di una matrice ridotta per righe sono linearmente indipendenti e le colonne che contengono pivot sono linearmente indipendenti. \\
	Analogamente vale per la riduzione per colonne. 
\end{prop}

\begin{fatto}
	Tutte e sole le matrici $ A \in \mathrm{Mat}_{n \times n}(\K) $ che commutano con ogni matrice $ B \in  \mathrm{Mat}_{n \times n}(\K) $ sono multipli dell'identità. \[AB = BA, \ \forall B \in \mathrm{Mat}_{n \times n}(\K) \quad \iff \quad \exists \lambda \in \K : A = \lambda \Id\]
\end{fatto}



\clearpage

\section{Applicazioni lineari}
% APPLICAZIONI LINEARI

\begin{definition}[Spazio delle funzioni]
	Sia $ A $ e $ V $ un $ \K $-spazio vettoriale. L'insieme $ V^A = \mathscr{F}{(A, V)} = \{f \colon A \to V\} $ con le operazioni di
	\begin{itemize}
		\item \emph{somma} $ \forall f, g \in \mathscr{F}(A, V) $: $ \forall x \in A, (f + g)(x) = f(x) + g(x) $;
		\item \emph{prodotto per scalare} $ \forall f \in \mathscr{F}(A, V), \forall \alpha \in \K $: $ \forall x \in A, (\alpha \cdot f)(x) = \alpha \cdot f(x) $
	\end{itemize} 
	è lo spazio vettoriale delle funzioni da $ A $ in $ V $. 
\end{definition}

\begin{definition}[Applicazione lineare]
	Siano $ V $ e $ W $ due $ \K $-spazi vettoriali. Diremo che una funzione $ L \colon V \to W  $ è un'applicazione lineare (o mappa lineare o omomorfismo) se $ L $ soddisfa le seguenti proprietà
	\begin{enumerate}[label = (\roman*)]
		\item $ \forall v_1, v_2 \in V $ vale $ L(v_1 + v_2) = L(v_1) + L(v_2) $
		\item $ \forall v \in V, \forall \lambda \in \K $ vale $ L(\lambda v) = \lambda L(v)$
	\end{enumerate}
	Diretta conseguenza è la seguente proprietà chiave delle applicazioni lineari
	\begin{enumerate}[resume, label = (\roman*)]
		\item $ L(O_V) = O_W $.
	\end{enumerate}
	L'insieme delle applicazioni lineari $ \mathscr{L}{(V, W)} = \{L \colon V \to W : L \text{ è lineare}\} $ è un sottospazio vettoriale di $ \mathscr{F}{(V, W)} $
\end{definition}

\begin{definition}[Nucleo]
	Siano $ V $ e $ W $ due spazi vettoriali su un campo $ \K $ e sia $ L \colon V \to W  $ un'applicazione lineare. Definiamo nucleo o kernel di $ L $, e scriveremo $ \ker{L}$, l'insieme degli elementi di $ V $ la cui immagine attraverso $ L $ è lo zero di $ W $. Formalmente \[\ker{L} = \{v \in V \colon L(v) = O_W\} \]
\end{definition}

\begin{prop}[di nucleo e immagine]
	Valgono le seguenti proprietà:
	\begin{enumerate}[label = (\roman*)]
		\item Sia $ L \in \mathscr{L}{(V, W)} $. Allora $ \ker{L} $ è un sottospazio vettoriale di $ V $.
		\item Sia $ L \in \mathscr{L}{(V, W)} $. Allora $ \im{L} $ è un sottospazio vettoriale di $ W $.
			Più in generale se $U$ è un sottospazio di $V$ allora $ \im{L|_U} $ è un sottospazio di $W$.
	\end{enumerate}
\end{prop}

\begin{prop}
	$ L \in \mathscr{L}{(V, W)} $ è iniettiva se e solo se $ \ker{L} = \{O_V\} $.
\end{prop}

\begin{prop}[Composizione di applicazioni lineari]
	La composizione di due applicazioni lineari è ancora un'applicazione lineare.
\end{prop}

\begin{thm}[Inversa di un'applicazione lineare]
	L'inversa di una applicazioni lineare (se esiste) è ancora un'applicazione lineare
\end{thm}

\begin{thm}
	Sia $ L \colon \K^n \to \K^n $ un'applicazione lineare tale che $ \ker{L} = \{O_V\} $. Se $ v_1, \ldots, v_n \in V $ sono vettori linearmente indipendenti, anche $ L(v_1), \ldots, L(v_n) $ sono vettori linearmente indipendenti di $ W $.
\end{thm}

\begin{thm}[delle dimensioni]
	Siano $ V $ e $ W $ spazi vettoriali su un campo $ \K $ e sia $ L \colon V \to W $  un'applicazione lineare. Allora vale \[\dim{V} = \dim{\im{L}} + \dim{\ker{L}}\]
\end{thm}

\begin{corollary}
	Valgono le seguenti relazioni:\\
	contenuto...
\end{corollary}

\begin{thm}[Biettività applicazioni lineari]
	Sia $ L \colon \K^n \to \K^n $ un'applicazione lineare. Se $ \ker{L} = \{O_V\} $ e $ \im{L} = W $, allora $ L $ è biettiva e dunque invertibile
\end{thm}

\begin{definition}[Isomorfismo]
	$ L \in \mathscr{L}{(V, W)} $ biettiva si dice isomorfismo. Se tale applicazioni esiste si dice che $ V $ e $ W $ sono isomorfi.
\end{definition}

\begin{definition}[Endomorfismo]
	$ L \in \mathscr{L}{(V, V)} $ dallo spazio in sé si dice endomorfismo. L'insieme degli endomorfismi viene indicato con $ \End{(V)} $ ed è un sottospazio vettoriale di $ \mathscr{F}{(V, V)} $
\end{definition}

\begin{prop}[Invertibilità endomorfismi]
	Un'endomorfismo è invertibile se e solo se è iniettivo.
\end{prop}

\begin{thm}[Applicazioni lineari e basi 1]
	Sia $V$ uno spazio vettoriale e $ \mathscr{B} = \{v_1, \ldots, v_n\} $ una base. Sia $ L \in \mathscr{L}{(V, W)} $ un'applixazione lineare. Allora $L$ è determinata dalla conoscenza di $L(v_i)$ per tutti gli elementi della base.
\end{thm}

\begin{thm}[Applicazioni lineari e basi 2]
	Siano $V$ e $W$ due spazi vettoriali, $ \mathscr{B} = \{v_1, \ldots, v_n\} $ una base di $V$ e $w_1, \ldots, w_n$ dei vettori di $W$. Esiste un'unica applicazione lineare $L \in \mathscr{L}{(V, W)}$ tale che $L(v_i) = w_i$.
\end{thm}

% Altri fatti di vaga utilità

\begin{fatto}[decomposizione di Fitting]
	Sia $ V $ un $ \K $-spazio vettoriale di $ \dim{V} = n $ e $ f \in \End{(V)} $. Allora esiste un intero $ k \leq n $ tale che \begin{enumerate}[label = (\roman*)]
		\item $ \ker{f^{k}} = \ker{f^{k + 1}} $;
		\item $ \im{f^{k}} = \im{f^{k + 1}} $;
		\item $ f\lvert_{\im{f^{k}}} \colon \im{f^{k}} \to \im{f^{k}} $ è un isomorfismo;
		\item $ f(\ker{f^{k}}) \subseteq \ker{f^{k}} $;
		\item $ f\lvert_{\ker{f^{k}}} \colon \ker{f^{k}} \to \ker{f^{k}} $ è nilpotente;
		\item $ V = \ker{f^{k}} \oplus \im{f^{k}} $.
	\end{enumerate}
\end{fatto}

\begin{fatto}
	Sia $ V $ un $ \K $-spazio vettoriale di $ \dim{V} = n $ e $ f \in \End{(V)} $. Allora:
	\begin{enumerate}
		\item $ \forall j \in \N, \ \ker{f^{j}} \subseteq \ker{f^{j + 1}} $;
		\item se esiste $ j \in \N : \ker{f^{j}} = \ker{f^{j + 1}} $ allora $ \forall m \geq j, \ \ker{f^{m}} = \ker{f^{m + 1}} $;
		\item se esiste $ j \in \N : f^{j} = 0 $ (endomorfismo nullo), allora $ f^{n} = 0 $. 
	\end{enumerate}
\end{fatto}

\clearpage

\section{Applicazioni lineari e matrici}
% APPLICAZIONI LINEARI E MATRICI

\begin{thm}
	Sia $ L \colon \K^n \to \K $ un'applicazione lineare. Allora esiste un unico vettore $ A \in \K^n $ tale che $ \forall X \in \K^n $ \[L(X) = A \cdot X\]
\end{thm}

\begin{thm}
	Esiste una corrispondenza biunivoca tra \[\{L \colon \K^n \to \K^m \quad \mathrm{lineare}\} \leftrightarrow \mathrm{Mat}_{m \times n}(\K)\]
	\begin{enumerate}
		\item Data una matrice $ M \in \mathrm{Mat}_{m \times n}(\K) $ è possibile associare ad essa un'applicazione lineare
		\begin{align*}
		L \colon \K^n & \to \K^m \\
		X & \mapsto MX
		\end{align*}
		\item Sia $ L \colon \K^n \to \K^m $ un'applicazione lineare. Allora esiste una matrice $ M \in \mathrm{Mat}_{m \times n}(\K) $ tale che $ \forall X \in \K^n $, $ L(X) = MX $. Se $ \mathscr{C} = \{e_1, \ldots , e_n\} $ la è base canonica di $ \K^n $, le colonne di $ M $ sono $ L(e_1), \ldots, L(e_n) $.
	\end{enumerate}
\end{thm}

\begin{thm}
	Siano $ V $ e $ W $ spazi vettoriali su un campo $ \K $ e sia $ L \colon \K^n \to \K^m $ un'applicazione lineare. Siano inoltre $ \mathscr{B} = \{v_1, \ldots, v_n\} $ e $ \mathscr{B}' = \{w_1, \ldots, w_m\} $ basi rispettivamente di $ V $ e di $ W $. Preso $ v \in V $ esiste una matrice $ M \in \mathrm{Mat}_{m \times n}(\K) $ tale che \[X_{\mathscr{B} '} (L(v)) = M X_{\mathscr{B}} (v) \] dove $ X(w) $ è il vettore colonna di $ w $ scritto nella rispettiva base. 
\end{thm}

\begin{thm}
	Siano $ V $ e $ W $ due spazi vettoriali su $ \K $ di dimensione $ n $ e $ m $. Siano $ \mathscr{B} = \{v_1, \ldots, v_n\} $ e $ \mathscr{B}' = \{w_1, \ldots, w_m\} $ basi rispettivamente di $ V $ e di $ W $. \\ Indicando con $ \mathscr{L}(V, W) = \{L \colon \K^n \to \K^m \quad \mathrm{lineare}\} $ e con $ \left [L \right ]_{\mathscr{B}'} ^{\mathscr{B}} $ la matrice associata a $ L $ rispetto alle basi $ \mathscr{B} $ e $ \mathscr{B}' $, si ha che 
	\begin{align*}
	M \colon \mathscr{L}(V, W) & \to \mathrm{Mat}_{m \times n}(\K)\\
	L & \mapsto \left [L \right ]_{\mathscr{B}'} ^{\mathscr{B}}
	\end{align*}
	è un'applicazione lineare ed è un isomorfismo tra lo spazio delle applicazioni lineari e lo spazio delle matrici. 
\end{thm}

\begin{thm}[matrice di funizione composta]
	Siano $ V $, $ W $ e $ U $ spazi vettoriali e siano \linebreak $ \mathscr{B} = \{v_1, \ldots, v_n\} $, $ \mathscr{B}' = \{w_1, \ldots, w_m\} $ e $ \mathscr{B}'' = \{u_1, \ldots, u_s\} $ basi di $ V $, $ W $ e $ U $ rispettivamente. Siano inoltre $ F \colon V \to W $ e $ G \colon W \to U $ lineari. Allora \[ \left [G \circ F \right ]_{\mathscr{B}''} ^{\mathscr{B}} = \left [G \right ]_{\mathscr{B}''} ^{\mathscr{B}'} \left [F \right ]_{\mathscr{B}'} ^{\mathscr{B}} \]
\end{thm}

\begin{thm}
	Sia $ L \colon V \to W $ un'applicazione lineare e $ B \colon V \to V $ lineare e invertibile. Allora vale che \[\im{L \circ B} = \im{L} \quad \mathrm{e} \quad \dim{\ker{L \circ B}} = \dim{\ker{L}}\] In altre parole se $ \left [L \right ] $ è la matrice associata a $ L $ e $ \left [B \right ] $ è la matrice invertibile delle mosse di colonna associata a $ B $, la matrice $ \left [L \right ] \left [B \right ] $ è una matrice ridotta a scalini per colonna in cui lo $ Span $ delle colonne è lo stesso dello span delle colonne di $ \left [L \right ] $. Più brevemente la riduzione di Gauss per colonne lascia invariato lo $ Span $ delle colonne. 
\end{thm}

\begin{thm}
	Sia $ L \colon V \to W $ un'applicazione lineare e $ U \colon W \to W $ lineare e invertibile. Allora vale che \[\ker{U \circ L} = \ker{L} \quad \mathrm{e} \quad \dim{\im{U \circ L}} = \dim{\im{L}}\] In altre parole se $ \left [L \right ] $ è la matrice associata a $ L $ e $ \left [U \right ] $ è la matrice invertibile delle mosse di riga associata a $ U $, la matrice $ \left [U \right ] \left [L \right ] $ è una matrice ridotta a scalini per riga che ha lo stesso $ \ker $ di $ \left [L \right ] $. Più brevemente la riduzione di Gauss per righe lascia invariato lo spazio delle soluzioni di un sistema lineare omogeneo.
\end{thm}

\begin{definition}[rango]
	Sia $ A \in \mathrm{Mat}_{m \times n} (\K) $ definiamo rango di A, $ \rg{A} $, in modo equivalente come
	\begin{enumerate}
		\item il numero massimo di colonne linearmente indipendenti (numero di \emph{pivot} colonna di $ A $ ridotta a scalini per colonna)
		\item il numero massimo di righe linearmente indipendenti (numero di \emph{pivot} riga di $ A $ ridotta a scalini per righe)
		\item la $ \dim{\im{L}} $, dove $ L \colon \K^n \to \K^m $ è l'applicazione lineare associata $ \left [L \right ]_{\mathscr{B}'} ^{\mathscr{B}} = A $
	\end{enumerate}
\end{definition}


\clearpage

\section{Sistemi lineari}
% SISTEMI LINEARI

\begin{definition}[Sistema lineare omogeneo]
	contenuto...
\end{definition}

\begin{definition}[Matrice associata al sistema lineare omogeneo]
	contenuto...
\end{definition}

\begin{prop}[Dimensione delle soluzioni]
	Sia $ M $ la matrice associata ad un sistema lineare omogeneo con $ n $ incognite. Indicando con $ S $ lo spazio delle soluzioni del sistema lineare vale \[\dim{S} = n - \rg{M}\]
\end{prop}

%	SUPERFLUO...
%\begin{definition}[sottospazio ortogonale]
%	contenuto... + \textsf{il sottospazio ortogonale è l'inieme delle soluzioni del sistema omogeneo}
%\end{definition}

\begin{definition}[Sistema lineare non omogeneo]
	contenuto...
\end{definition}

\begin{definition}[Matrice completa e incompleta associata]
	contenuto...
\end{definition}

\begin{thm}[Rouchè-Capelli]
	Un sistema omogeneo ammette soluzione se e solo se il rango della matrice completa e incompleta coincidono.
\end{thm}

\begin{thm}[Insieme soluzioni del sistema non omogeneo]
	Sia $ S $ l'insieme delle soluzioni del sistema non omogeneo e $ S_0 $ l'insieme delle soluzioni del sistema omogeneo associato. Supposto $ S \neq \emptyset $, preso un qualunque $ v \in S $ vale \[S = v + S_0\]
\end{thm}

\begin{definition}[Sottospazio affine]
	Sia $ V $ un $ \K $-spazio vettoriale, $ U $ un suo sottospazio e $ v \in V - U $ ($ v \neq 0 $) si dice che l'insieme $ v + U $ è un sottospazio affine di $ V $. Per convenzione si pone $ \dim{(v + U)} = \dim{U} $.
\end{definition}

\clearpage

\section{Determinante}
% 	DETERMINIANTE

\begin{definition}[gruppo simmetrico]
	Il gruppo simmetrico di un insieme è il gruppo formato dall'insieme delle permutazioni dei suoi elementi, cioè dall'insieme delle funzioni biiettive di tale insieme in se stesso, munito dell'operazione binaria di composizione di funzioni. \\
	In particolare detto $ S_n = \{1, \ldots, n\} $ l'insieme delle permutazioni \[\Sigma_n = \Sigma (S_n) = \{\sigma \colon S_n \to S_n : \sigma \text{ è biettiva}\}\] è un gruppo simmetrico. Ricordiamo che una permutazione $ \sigma \in \Sigma_n $ viene spesso indicata con la seguente notazione 
	\[\begin{pmatrix}
	1 & 2 & \cdots & n \\
	\sigma (1) & \sigma (2) & \cdots & \sigma (n) \\
	\end{pmatrix}\]
	dove si intende che l'elemento $ i $ viene mandato in $ \sigma (i) $ dalla permutazione. 
\end{definition}

\begin{definition}[trasposizione]
	Una trasposizione è una permutazione $ \tau \in \Sigma_n $ tale che scambia due soli elementi di $ S_n $ mentre lascia invariati i restaniti $ n - 2 $. Se $ \tau $ scambia $ i, j \in S_n $ scriveremo $ \begin{pmatrix}
	i & j \\
	\end{pmatrix} $. 
\end{definition}

\begin{prop}
	\begin{enumerate}[label = (\roman*)]
		\item Ogni permutazione $ \sigma \in \Sigma_n $ è esprimile, non in modo unico, come prodotto (composizione) di trasposizioni. 
		\item Se $ \sigma = \tau_1 \circ \cdots \circ \tau_h = \lambda_1 \circ \cdots \circ \lambda_h $ con $ \tau_i $ e $ \lambda_j $ trasposizioni allora $ h $ e $ k $ hanno la stessa parità. Se $ \sigma $ è prodotto di un numero pari (dispari) di trasposizioni diremo che $ \sigma $ è pari (dispari). 
	\end{enumerate}
\end{prop}

\begin{definition}
	Data $ \sigma \in \Sigma_n $ definiamo la funzione segno $ \sgn \colon \Sigma_n \to \{-1, 1\} $ come 
	\[\sgn(\sigma) = 
	\begin{cases*}
	1 & se $ \sigma $ è pari \\
	-1 & se $ \sigma $ è dispari \\
	\end{cases*}\]
	Vale in particolare che $ \sgn(\sigma_1 \circ \sigma_2) = \sgn(\sigma_1) \cdot \sgn(\sigma_2) $
\end{definition}

\begin{prop}
	Sia $ \sigma \in \Sigma_n $ una permutazione e sia $ \sigma^{-1} $ la permutazione inversa. Allora $ \sgn(\sigma) = \sgn(\sigma^{-1}) $. 
\end{prop}

\begin{thm}[Unicità del determinante]
	Sia $ \mathrm{Mat}_{n \times n} (\K) $  lo spazio vettoriale delle matrici quadrate  a valori nel campo $ \K $. Esiste una ed una sola funzione da $ \mathrm{Mat}_{n \times n} (\K) $ in $ \K $ funzione delle righe (o delle colonne) di una matrice $ A $ che rispetta i seguenti tre assiomi:  
	\begin{enumerate}[label = (\roman*)]
		\item \emph{multilineare} (lineare in ogni riga o colonna);
		\item \emph{alternante} (cambia di segno se si scambiano due righe o due colonne);
		\item \emph{normalizzata} (l'immagine dell'identità è 1);
	\end{enumerate}
	Tale funzione viene detta determinante ed indicata con $ \det \colon \mathrm{Mat}_{n \times n} (\K) \to \K $. 
\end{thm}

\begin{propriety}[del determinante] \label{prop:det}
	Le seguenti proprietà sono conseguenza degli assiomi (i), (ii) e (iii). Sia $ A \in \mathrm{Mat}_{n \times n} (\K) $ allora
	\begin{enumerate}[label = (\arabic*)]
		\item Se $ A $ ha due righe uguali allora $ \det{A} = 0 $.
		\item Se $ A $ ha una riga nulla allora $ \det{A} = 0 $.
		\item Se alla riga $ A_i $ di $ A $ si somma un multiplo della riga $ A_j $ ($ i \neq j $) si ottiene una matrice $ B $ tale che $ \det A = \det B $. 
		\item Il determinante è invariante sotto l'algoritmo di Gauss (escludendo le mosse di \emph{normalizzazione} delle righe o delle colonne) a meno di un segno che dipende dal numero di scambi di righe o di colonne fatto. In altre parole se $ S $ è una forma a scalini di $ A $ allora $ \det A = \pm \det S $.
		\item Se $ A $ è una matrice diagonale allora il suo determinante è il prodotto degli elementi sulla diagonale: $ \det A = a_{11} \cdots a_{nn} $. 
	\end{enumerate}
\end{propriety}

\begin{thm}[esistenza del determinante]
	Sia $ A = (a_{ij})_{\substack{i = 1, \ldots, n \\ j = 1, \ldots, n}} \in \mathrm{Mat}_{n \times n} (\K) $. La funzione \[\det(A) = \sum_{\sigma \in \Sigma_n} \sgn(\sigma) \cdot a_{1 \sigma(1)} a_{2 \sigma(2)} \cdots a_{n \sigma{n}}\] è il determinante (in quanto è una funzione multilineare, alternante e normalizzata dallo spazio delle matrici nel campo). 
\end{thm}

\begin{corollary}
	Il determinante di $ A $ è uguale al determinante della sua trasposta: $ \det A = \det A^{t} $. 
\end{corollary}

\begin{definition}[complemento algebrico]
	Il complemento algebrico o cofattore dell'elemento $ a_{ij} $ di una matrice $ A \in \mathrm{Mat}_{n \times n} (\K) $ è il determinante della matrice $ (n - 1) \times (n - 1) $ ottenuta cancellando da $ A $ la $ i $-esima riga e la $ j $-esima colonna moltiplicato per $ (-1)^{i + j} $: in formule
	\[\cof_{ij}(A) = (-1)^{i + j} \cdot \det 
	\begin{pmatrix}
	a_{11} & \cdots & \cancel{a_{1j}} & \cdots & a_{1n} \\
	\vdots &        & \vdots &        & \vdots \\
	\cancel{a_{i1}} & \cdots & \cancel{a_{ij}} & \cdots & \cancel{a_{in}} \\
	\vdots &  		& \vdots &  	  & \vdots \\
	a_{n1} & \cdots & \cancel{a_{nj}} & \cdots & a_{nn} \\
	\end{pmatrix}.\]
	Con $ \cof{(A)} $ indichiamo la matrice dei cofattori ovvero la matrice che ha nella posizione $ i, j $ il complemento algebrico di $ a_{ij} $, $ \cof{(A)} = \left(\cof_{ij}{(A)}\right) $
\end{definition}

\begin{thm}[sviluppo di Laplace]
	Data $ A \in \mathrm{Mat}_{n \times n} (\K) $ la seguente funzione
	\begin{itemize}
		\item fissata una riga $ i $ di $ A $: $ \det A = \sum_{j = 1}^{n} a_{ij} \cdot \cof_{ij} (A) $
		\item fissata una colonna $ j $ di $ A $: $ \det A = \sum_{i = 1}^{n} a_{ij} \cdot \cof_{ij} (A) $
	\end{itemize}
	verifica gli assiomi (i), (ii) e (iii) e quindi è il determinante. \\
	Dallo sviluppo di Laplace si deduce che la proprietà (5) di Proprietà \ref{prop:det} vale anche per le matrici triangolari (superiori o inferiori)
\end{thm}

\begin{thm}[invertibilità]
	$ A $ è invertibile $ \iff $ $ \det{A} \neq 0 $ ($ \iff $ $ \rg A = n $).
\end{thm}

\begin{prop}[formula per l'inversa]
	Sia $ A \in \mathrm{Mat}_{n \times n} (\K) $ invertibile, i.e. $ \det A \neq 0 $. Allora il coefficiente $ ij $ della matrice inversa è \[\left(A^{-1}\right)_{ij} = \frac{1}{\det{A}} \cdot \cof_{ji}{(A)}\] dove $ \cof_{ji}{(A)} $ è il complemento algebrico dell'elemento $ a_{ji} $ di $ A $ (sì, gli indici sono scambiati).   
\end{prop}

\begin{thm}[regola di Cramer]
	Sia $ A \in \mathrm{Mat}_{n \times n} (\K) $ invertibile, i.e $ \det A \neq 0 $ (e $ \rg A = n $), e siano $ A^{1}, \ldots, A^{n} $ le sue colonne. Siano inoltre $ b = (b_j) $ un vettori colonna. Allora se $ x = (x_j) $ è l'unico vettore colonna che soddisfa il sistema lineare 
	\[Ax = b \quad \iff \quad A \begin{pmatrix} x_1 \\ \vdots \\ x_n \end{pmatrix} = \begin{pmatrix} b_1 \\ \vdots \\ b_n \end{pmatrix} \quad \iff \quad x_1 A^{1} + \ldots x_n A^{n} = b\]
	ha componenti date da 
	\[x_j = \frac{\det{\left(A^{1} \cdots A^{j - 1} \ b \ A^{j + 1} \ldots A^{n}\right)}}{\det{A}}\]
	dove per $ A^{1} \cdots A^{j - 1} \ b \ A^{j + 1} \ldots A^{n} $ si intende la matrice $ A $ alla cui $ j $-esima colonna è stato sostituito il vettore colonna $ b $ dei termini noti.  
\end{thm}

\begin{corollary}
	Sia $ A \in \mathrm{Mat}_{n \times n} (\K) $ e $ \cof{A} $ la matrice dei cofattori. Allora vale la seguente identità \[A \ (\cof{A})^{t} = \det{(A)} \cdot \Id\]
\end{corollary}

\begin{thm}[di Binet]
	Date $ A, B \in \mathrm{Mat}_{n \times n} (\K) $ vale $ \det{(AB)} = \det{(A)} \cdot \det{(B)} $. 
\end{thm}

\begin{corollary}[determinante dell'inversa]
	Se $ A \in \mathrm{Mat}_{n \times n} (\K) $ è invertivile, i.e. $ \det{A} \neq 0 $, allora $ \det{A^{-1}} = \frac{1}{\det{A}} $. 
\end{corollary}

\begin{corollary}
	Per ogni $ A, B \in \mathrm{Mat}_{n \times n} (\K) $ vale $ \det{AB} = \det{BA} $.
\end{corollary}

\begin{corollary}[invarianza del determinatane per coniugio]
	Siano $ A, B \in \mathrm{Mat}_{n \times n} (\K) $ con $ B $ matrice invertibile. Allora $ \det{(B^{-1} A B)} = \det{(A)} $. \\
	In altri termini se $ [A] $ e $ [A'] $ sono matrici che descrivono lo stesso endomorfismo $ A \colon V \to V $ ma scritte in basi (in partenza ed in arrivo) diverse allora $ \det{[A]} = \det{[A']} $. Dunque è il determinante è ben definito come funzione dagli endomorfismi $ \mathscr{L}(V) $ nel campo $ \K $ . 
\end{corollary}

\begin{definition}[sottomatrice]
	Sia $ M \in \mathrm{Mat}_{n \times m} (\K) $ una matrice qualsiasi. Per sottomatrice di $ M $ si intende una ottenuta da $ M $ cancellando alcune righe e/o alcune colonne di $ M $. In modo equivalente si intende una $ M' \in \mathrm{Mat}_{r \times s} (\K) $ ottenuta da $ M $ selezionando i coefficienti posti nell'intersezione tra $ 1 \leq r \leq n $ righe ed $ 1 \leq s \leq m $ colonne scelte nella matrice $ M $. Nel caso di sottomatrice quadrate si ha $ r = s = k $ e si dice che l'ordine di $ M' $ è $ k $. 
\end{definition}

\begin{prop}
	Se $ v_1 = \begin{pmatrix} v_{11} \\ v_{n1} \end{pmatrix}, \ldots, v_k = \begin{pmatrix} v_{1k} \\ v_{nk} \end{pmatrix} $ con $ k \leq n $ sono vettori linearmente \emph{dipendenti} allora ogni sottomatrice quadrata di ordine $ k $ estratta dalla matrice $ M \in \mathrm{Mat}_{n \times k} (\K) $ che ha per colonne $ v_1, \ldots, v_k $ non invertibile e quindi con determinante nullo. 
	\[M = \begin{pmatrix}[c|c|c]
	v_{11} & & v_{1k} \\
	\vdots & \cdots & \vdots \\
	v_{n1} & & v_{nk}
	\end{pmatrix}\]
\end{prop}

\begin{prop}
	Se $ v_1 = \begin{pmatrix} v_{11} \\ v_{1n} \end{pmatrix}, \ldots, v_k = \begin{pmatrix} v_{k1} \\ v_{kn} \end{pmatrix} $ con $ k \leq n $ sono vettori linearmente \emph{indipendenti} allora esiste una sottomatrice quadrata di ordine $ k $ estratta dalla matrice $ M \in \mathrm{Mat}_{n \times k} (\K) $ che ha per colonne $ v_1, \ldots, v_k $ invertibile e quindi con determinante non nullo. 
	\[M = \begin{pmatrix}[c|c|c]
	v_{11} & & v_{1k} \\
	\vdots & \cdots & \vdots \\
	v_{n1} & & v_{nk}
	\end{pmatrix}\]
\end{prop}

\begin{thm}[caratterizzazione del rango con il determinante]
	Sia $ A \in \mathrm{Mat}_{n \times n} (\K) $. Allora il rango di $ A $ è il massimo ordine di una sottomatrice quadrata invertibile, i.e con determinante non nullo. \\
	In altri termini $ \rg A = k $ $ \iff $ tra tutte le sottomatrici di $ A $ esiste una sottomatrice $ k \times k $ con determinante $ \neq 0 $ tale che tutte le sottomatrici quadrate di ordine maggiore hanno determinante nullo. Per lo sviluppo di Laplace è sufficiente verificare che tutte le sottomatrici $ (k + 1) \times (k + 1) $ hanno determinante nullo. 
\end{thm}

\begin{propriety}[del determinante]
	Valgono le seguenti identità aggiuntive. 
	\begin{enumerate}
		\item Se $ M = \begin{pmatrix}
		A & B \\
		C & D
		\end{pmatrix} $	
		è una matrice a blocchi con $ A $ matrice invertibile allora $ \det{M} = \det{(A)} \cdot \det{(D - CA^{-1}B)} $. Inoltre che $ AC = CA $ allora $ \det{M} = \det{AD - CB} $. 
		\item Se $ M $ è una matrice diagonale a blocchi allora il determinante è il prodotto dei determinanti dei blocchi diagonali $ A_1, \ldots A_k $. 
		\[\det{M} = \det
		\setlength{\arraycolsep}{0pt}
		\begin{pmatrix}
		\fbox{$ A_1 $}		&		&		&		&		 \\
		& \fbox{$ A_2 $}	&		&		&		 \\
		& 		& \ \ddots \ &		&		 \\	 
		& 		&		& \fbox{$ A_{k - 1} $}		&		 \\	
		& 		& &		& \fbox{$ A_k $}		 \\	 		 		
		\end{pmatrix}
		= \det{(A_1)} \cdots \det{(A_k)}\]
		\item Se $ M $ è una \emph{matrice di Vandermonde} allora vale che 
		\[\det{(M)} = \det 
		\begin{pmatrix}
		1 & 1 & \cdots & 1 \\
		a_1 & a_2 & \cdots & a_n \\
		a_1^{2} & a_2^{2} & \cdots & a_n^{2} \\
		\vdots & \vdots & & \vdots \\
		a_1^{n} & a_2^{n} & \cdots & a_n^{n} \\
		\end{pmatrix} 
		= \prod_{\substack{i = 1 \\ j < i}}^{n} (a_i - a_j)\]
	\end{enumerate}
\end{propriety}

\clearpage

\section{Diagonalizzazione di endomorfismi}
% DIAGONALIZZAZIONE DI ENDOMORFISMI

\begin{definition}[autovettore, autovalore]
	Sia $ V $ un $ \K $-spazio vettoriale. Sia $ T \colon V \to V $ un endomorfismo su $ V $. Un vettore $ v \in V \setminus \{O_V\} $ si dice autovettore di $ T $ se esiste $ \lambda \in \K $ tale che \[T(v) = \lambda v.\] Si dice in questo caso che $ \lambda $ è autovalore per $ T $ (relativo a $ v $). \\
	Notiamo che tutti i $ v \in \ker{V} \setminus \{O_V\} $ sono autovettori per $ v $ con autovalore 0.  
\end{definition}

\begin{definition}[autospazio]
	Dato $ \lambda \in \K $ chiamiamo $ V_\lambda = \{v \in V : T(v) = \lambda v\} $ l'autospazio relativo a $ \lambda $. Segue dalla definizione che $ V_\lambda = \ker{\left(T - \lambda \cdot \Id\right) }$
\end{definition}

\begin{prop}
	Se $ v_1, \ldots, v_n \in V $ sono autovettori per $ T $ con relativi autovalori $ \lambda_1, \ldots, \lambda_n \in \K $ e $ \mathscr{B} = \{v_1, \ldots, v_n\} $ è una base di $ V $ allora la matrice di $ T \in \End{(V)} $ rispetto a $ \mathscr{B} $ (sia in partenza che in arrivo) è diagonale. 
	\[[T]_{\mathscr{B}}^{\mathscr{B}} = 
	\begin{pmatrix}
	\lambda_1 & \cdots & 0 \\
	\vdots & \ddots & \vdots \\
	0 & \ldots & \lambda_n \\
	\end{pmatrix}\]
\end{prop}

\begin{definition}[polinomio caratteristico]
	Sia $ T \in \End{(V)} $. Fissata una base $ \mathscr{B} = \{v_1, \ldots, v_n\} $ di $ V $ chiamiamo \[p_T(t) = \det \left(t \cdot \Id - [T]_{\mathscr{B}}^{\mathscr{B}}\right)\] il polinomio caratteristico di $ T $. 
\end{definition}

\begin{prop}
	Il polinomio caratteristico non dipende dalla base scelta. In altri termini se $ \mathscr{B} = \{v_1, \ldots, v_n\} $ e $ \mathscr{B}' = \{v'_1, \ldots, v'_n\} $ sono basi di $ V $ allora \[p_T(t) = \det \left(t \cdot \Id - [T]_{\mathscr{B}}^{\mathscr{B}}\right) = \det \left(t \cdot \Id - [T]_{\mathscr{B}'}^{\mathscr{B}'}\right) = \det \left(t \cdot \Id - T\right)\]
\end{prop}

\begin{thm}
	Sia $ T \in \End{(V)} $. Allora $ \lambda \in \K $ è un autovalore di $ T $ se e solo se $ \lambda $ è radice di $ p_T(t) $, ossia $ p_T(\lambda) = 0 $. 
\end{thm}

\begin{thm}
	Dati $ \lambda_1, \ldots, \lambda_k $ autovalori di $ T \in \End{(V)} $ a due a due distinti, siano $ v_1, \ldots, v_k $ gli autovettori corrispondenti: $ T(v_1) = \lambda_1 v_1, \ldots, T(v_k) = \lambda_k v_k $. Allora $ v_1, \ldots, v_k $ sono linearmente indipendenti. 
\end{thm}

\begin{thm}[somma diretta degli autospazi]
	Sia $ T \in \End{(V)} $ e $ \lambda_1, \ldots, \lambda_k $ autovalori di $ T $ a due a due distinti. Allora gli autospazi $ V_{\lambda_1}, \ldots, V_{\lambda_k} $ sono in somma diretta. \\
	Più in generale se $ A_1, \ldots, A_k $ sottospazi di $ V $ sono tali che per ogni insieme $ \{v_1, \ldots, v_k\} $ di vettori non nulli linearmente indipendenti tali che $ v_i \in A_i $, allora $ A_1, \ldots, A_k $ sono in somma diretta. 
\end{thm}

\begin{definition}[molteplicità algebrica e geometrica]
	Sia $ T \in \End{(V)} $ e \[p_T(t) = (t - \lambda_1)^{a_1} \cdots (t - \lambda_k)^{a_k} \cdot f(t)\] il polinomio caratteristico dove $ \lambda_1, \ldots, \lambda_k $ sono le radici del polinomio, i.e autovalori di $ T $, e $ f(t) $ un polinomio irriducibile in $ \K[t] $. Diremo che $ a_i $ è la molteplicità algebrica dell'autovalore $ \lambda_i $. Diremo inoltre che $ m_i = \dim{V_{\lambda_i}} $ è la molteplicità geometrica di $ \lambda_i $. \\
	Notiamo che se $ T $ è diagonalizzabile allora $ f(t) = 1 $. 
\end{definition}

\begin{thm}
	Sia $ T \in \End{(V)} $ e $ \lambda_1, \ldots, \lambda_k $ con $ k \leq n $ autovalori. Allora $ \forall i = 1, \ldots, k $ vale $ 1 \leq m_i \leq a_i $ (molteplicità geometrica $ \leq $ molteplicità algebrica). 
\end{thm}

\begin{corollary}[criterio sufficiente per la diagonalizzazione]
	Sia $ T \colon V \to V $ un endomorfismo e $ p_T(t) $ il suo polinomio caratteristico. Se $ p_T(t) $ ha tutte le radici in $ \K $ a due a due distinte allora $ T $ è diagonalizzabile. 
\end{corollary}

\begin{thm}
	Sia $ T \in \End{(V)} $. Allora $ T $ è diagonalizzabile se e solo se $ f(t) = 1 $ (il polinomio si fattorizza completamente nel campo) e $ \forall \lambda_i $ autovalore $ m_i = a_i $. 
\end{thm}

\begin{definition}[polinomio minimo]
	Sia $ T \in \End{(V)} $. Chiamiamo polinomio minimo di $ T $ il polinomio di grado più piccolo (\emph{wlog} monico) $ \mu_T(t) \in \K[t] $ tale che \[\mu_T(T) = T^j + \ldots + b_1 T + b_0 \Id = 0.\] 
\end{definition}

\begin{thm}
	Sia $ T \in \End{(V)} $. Se $ h(t) \in \K[t] $ soddisfa la proprietà $ h(T) = 0 $ allora $ \mu_T(t) $ divide $ h(t) $. 
\end{thm}

\begin{thm}[di Hamilton - Cayley] \label{thm:HC}
	Dato $ T \colon V \to V $ endomorfismo vale che $ p_T(T) = 0 $ e quindi che il polinomio minimo divide il polinomio caratteristico. 
\end{thm}

\begin{prop}
	Sia $ A, B \in \mathrm{Mat}_{n \times n} (\K) $ con $ B $ invertibile. Allora se $ q(t) = c_n t^{n} + \ldots + c_1 t + c_0 $ è un polinomio in $ \K[t] $ vale \[q(B^{-1} A B) = B^{-1} q(A) B. \] Se $ T \in \End{(V)} $ e $ q(t) = p_T(t) $ è il polinomio caratteristico di $ T $ (o un qualsiasi polinomio tale che $ q(T) = 0 $) si ha che $ p_T(T) = 0 \iff p_T(B^{-1} T B) = 0 $ e quindi il polinomio minimo non dipende dalla base e l'enunciato del Teorema \ref{thm:HC} non dipende dalla base scelta per $ V $. 
\end{prop}

\begin{prop}
	$ T \in \End{(V)} $ è diagonalizzabile se e solo se le radici del polinomio minimo $ \mu_T(t) $ hanno molteplicità algebrica 1, ovvero $ \mu_T(t) $ non ha radici doppie. 
\end{prop}

\begin{prop}
	Sia $ T \in \End{(V)} $, $ p_T(t) $ il polinomio caratteristico e $ \mu_T(t) $ il polinomio minimo. Allora $ p_T(\lambda) = 0 \iff \mu_T(\lambda) = 0 $. 
\end{prop}

\begin{thm}[triangolazione]
	Data una qualunque matrice $ M \in \mathrm{Mat}_{n \times n} (\K) $ con $ \K $ algebricamente chiuso (i.e. ogni polinomio in $ \K[t] $ è irriducibile, per esempio $ \K = \C $) esiste una matrice $ C $ invertibile (matrice di cambio base) tale che $ CMC^{-1} $ è triangolare superiore. \\ Moralmente ogni matrice è triangolabile. 
\end{thm}

\begin{prop}
	Siano $ T, S \in \End{(V)} $ diagonalizzabili tali che $ TS = ST $. Allora $ S $ preserva gli autospazi di $ T $ e viceversa: se $ V_{\lambda} $ è autospazio di $ T $ allora $ S(V_{\lambda}) \subseteq V_{\lambda} $. 
\end{prop}

\begin{thm}[diagonalizzazione simultanea]
	Siano $ T, S \in \End{(V)} $ diagonalizzabili. Allora $ T $ e $ S $ sono simultaneamente diagonalizzabili (i.e. esiste una base che diagonalizza entrambe) se e solo se $ TS = ST $.
\end{thm}

\begin{thm}
	Sia $ V $ un $ \K $-spazio vettoriale, $ f \in \End{(V)} $ diagonalizzabile e $ W $ un sottospazio di $ V $. Allora \[W = (W \cap V_{\lambda_1}) \oplus \cdots \oplus (W \cap V_{\lambda_k})\] dove $ V_{\lambda_1}, \ldots, V_{\lambda_k} $ sono gli autospazi relativi agli autovalori $ \lambda_1, \ldots, \lambda_k $ di $ f $. Dunque la restrizione di $ f $ a $ W \cap V_{\lambda_j} $ è diagonalizzabile per ogni $ j $ e quindi $ f\lvert_{W} $ è diagonalizzabile.
\end{thm}

\begin{thm}[esistenza del complementare invariante]
	Sia $ V $ un $ \K $-spazio vettoriale, $ f \in \End{(V)} $ diagonalizzabile e $ W $ un sottospazio di $ V $ $ f $-invariante. Allora esiste $ U $ sottospazio vettoriale $ f $-invariante tale che $ V = W \oplus U $. 
\end{thm}

\begin{thm}
	Sia $ V $ un $ \K $-spazio vettoriale e $ f \in \End{(V)} $. Se esistono $ W_1, W_2 $ un sottospazi di $ V $ $ f $-invarianti tali che $ V = W_1 + W_2 $ e tali che le restrizioni di $ f $ a $ W_1 $ e $ W_2 $ sono diagonalizzabili, allora $ f $ è diagonalizzabile. 
\end{thm}

\begin{thm} \textbf{*}
	Sia $ V $ un $ \K $-spazio vettoriale e $ W, U $ due sottospazi di $ V $ tali che $ V = W \oplus U $. Se $ f \colon W \to W $ e $ g \colon U \to U $ sono due applicazioni lineari si consideri $ L \colon V \to V $ data da $ L(v) = f(w) + g(u) $, dove $ v = w + u $, $ w \in W $, $ u \in U $. Allora $ L $ è diagonalizzabile se e solo se $ f $ e $ g $ sono diagonalizzabili.
\end{thm}

\clearpage

\vspace{2mm}

\begin{framed}
	\begin{center}
		\textbf{Diagonalizzazione - una strategia in 4 passi}
	\end{center}
	\begin{enumerate}
		\item Dato $ T \colon V \to V $ endomorfismo, calcolo il polinomio caratteristico e ne trovo le radici ottenendo un polinomio della forma \[p_T(t) = \det{(t \Id - T)} = (t - \lambda_1)^{a_1} \cdots (t - \lambda_k)^{a_k} \cdot f(t)\] con $ f(t) $ irriducibile in $ \K[t] $. Se il polinomio caratteristico si fattorizza completamente nel campo, i.e. $ f(t) = 1 $ allora posso procedere, altrimenti $ T $ non è diagonalizzabile. 
		\item Dette $ \lambda_1, \ldots, \lambda_k $ con $ k \leq n $ le radici del polinomio caratteristico, i.e. autovalori di $ T $, studio gli autospazi relativi $ V_{\lambda_1} = \ker{(T - \lambda_1 \Id)}, \ldots, V_{\lambda_k} = \ker{(T - \lambda_k \Id)} $
		\item Osservo che questi sottospazi sono in somma diretta e quindi 
		\begin{itemize}
			\item se $ V_{\lambda_1} \oplus \ldots \oplus V_{\lambda_k} = V $ allora $ T $ si diagonalizza e una base digonalizzante di $ V $ è data dall'unione delle basi dei $ V_{\lambda_j} $ e posso procedere;
			\item se $ V_{\lambda_1} \oplus \ldots \oplus V_{\lambda_k} \subset V $ allora $ T $ non è diagonalizzabile.
		\end{itemize} 
		Se voglio solo sapere se $ T $ è diagonalizzabile è sufficiente confrontare la molteplicità algebrica $ a_i $ delle radici $ \lambda_i $ del polinomio caratteristico con la molteplicità geometrica $ m_i = \dim{V_{\lambda_i}} = \dim{\ker{(T - \lambda_i \Id)} }$. $ T $ è diagonalizzabile $ \iff $ per ogni $ i $ vale $ a_i = m_i $.
		\item Usando la base trovata, si scrive la matrice diagonale corrispondente (i coefficienti sono zeri tranne sulla diagonale in cui ci sono tanti $ \lambda_i $ quanti la $ m_i = \dim{V_{\lambda_i}} $). 
	\end{enumerate} 
\end{framed}


\clearpage

\section{Prodotti scalari}
% PRODOTTI SCALARI

\begin{definition}[prodotto scalare]
	Sia $ V $ un $ \K $-spazio vettoriale. Un prodotto scalare su $ V $ è una funzione \[\scprd{\;}{\,} \colon V \times V \to \K \] che soddisfa le seguenti prorpietà
	\begin{enumerate}[label=(\roman*)]
		\item $ \forall v,w \in V $ vale $ \scprd{v}{w} = \scprd{w}{v} $
		\item $ \forall v, w, u \in V $ vale $ \scprd{v}{w + u} = \scprd{v}{w} + \scprd{v}{u} $
		\item $ \forall v, w \in V, \forall \lambda \in \K $ vale $ \scprd{\lambda v}{w} = \scprd{v}{\lambda w} = \lambda \scprd{v}{w} $
	\end{enumerate}	
	Indicheremo con la coppia $ (V, \varphi) $ lo spazio vettoriale $ V $ dotato del prodotto scalare $ \varphi \colon V \times V \to \K $. 
\end{definition}

\begin{propriety} Alcuni prodotti scalari godono delle segenti proprietà
	\begin{enumerate}
		\item Un vettore $ v \in V $ tale che $ \scprd{v}{v} = 0 $ si dice \emph{isotropo}.
		\item Un prodotto scalare tale che preso un $ v \in V $ \[\forall w \in V \quad \scprd{v}{w} = O_V \Rightarrow v = O_V\] si dice \emph{non degenere}.
		\item Un prodotto scalare su $ V $ spazio vettoriale sul campo $ \R $ tale che \[ \forall v \in V \quad \scprd{v}{v} \geq 0 \quad \mathrm{e} \quad \scprd{v}{v} = 0 \iff v = O_V \] si dice \emph{definito positivo}.
	\end{enumerate}
\end{propriety}

\begin{thm}
	Un prodotto scalare è non degenere se e solo se la matrice $ E $ che lo rappresenta ha rango massimo. Ovvero, se $ \{e_1, \ldots, e_n \} $ è base di $ V $
	\[ \rg{E} = \rg{ 
		\begin{pmatrix}
		\scprd{e_1}{e_1} & \cdots  & \scprd{e_1}{e_n} \\
		\vdots           & \ddots & \vdots \\
		\scprd{e_n}{e_1} & \cdots  & \scprd{e_n}{e_n} \\
		\end{pmatrix}}
	= n	\]
\end{thm}

\begin{definition}[norma, distanza, ortogonalità]
	Sia $ \scprd{\,}{} $ un prodotto scalare su $ V $ spazio vettoriale su $ \R $. Allora
	\begin{enumerate}
		\item la norma di $ v \in V $ è $ \norm{v} = \sqrt{\scprd{v}{v}} $
		\item la distanza tra $ v, w \in V $ è data da $ \norm{v - w} $
		\item due vettori $ v, w \in V $ si dicono ortogonali se $ \scprd{v}{w} = 0 $
	\end{enumerate}
\end{definition}

\begin{thm}[di Pitagora]
	Sia $ V $ uno spazio vettoriale sul campo $ \R $ con prodotto scalare definito positivo. Dati $ v, w \in V \colon \scprd{v}{w} = 0 $	allora \[\norm{v + w}^2 = \norm{v}^2 + \norm{w}^2\]
\end{thm}

\begin{thm}[del parallelogramma]
	Sia $ V $ uno spazio vettoriale sul campo $ \R $ con prodotto scalare definito positivo. Allora $ \forall v, w \in V $ vale \[\norm{v + w}^2 + \norm{v - w}^2 = 2\norm{v}^2 + 2\norm{w}^2\]
\end{thm}

\begin{definition}[componente]
	Dati $ v, w \in V $ chiamiamo coefficiente di Fourier o componente di $ v $ lungo $ w $ lo scalare \[c = \frac{\scprd{v}{w}}{\scprd{w}{w}}\] In particolare vale $ \scprd{v - cw}{w} = 0 $
\end{definition}

\begin{thm}[disuguaglianza di Schwarz]
	Sia $ V $ uno spazio vettoriale sul campo $ \R $ con prodotto scalare definito positivo. Allora $ \forall v, w \in V $ vale \[|\scprd{v}{w}| \leq \norm{v} \cdot \norm{w}\]
\end{thm}

\begin{thm}[disuguaglianza triangolare]
	Sia $ V $ uno spazio vettoriale sul campo $ \R $ con prodotto scalare definito positivo. Allora $ \forall v, w \in V $ vale \[\norm{v + w} \leq \norm{v} + \norm{w}\]
\end{thm}

\begin{thm}
	Siano $ v_1, \ldots , v_n \in V $ a due a due perpendicolari ($ \forall i \neq j \; \scprd{v_i}{v_j} = 0 $). Allora $ \forall v \in V $ il vettore \[v - \frac{\scprd{v}{v_1}}{\scprd{v_1}{v_1}}v_1 - \ldots - \frac{\scprd{v}{v_n}}{\scprd{v_n}{v_n}}v_n\] è ortogonale a ciascuno dei $ v_i $. Risulta inoltre che il vettore $ c_1 v_1 + \ldots + c_n v_n $ (dove $ c_i = \frac{\scprd{v}{v_i}}{\scprd{v_i}{v_i}} $) è la migliore approssimazione di $ v $ come combinazione lineare dei $ v_i $.
\end{thm}

\begin{thm}[disuguaglianza di Bessel]
	Siano $ e_1, \ldots , e_n \in V $ a due a due perpendicolari e unitari. Dato un $ v \in V $, sia $ c_i = \frac{\scprd{v}{e_i}}{\scprd{e_i}{e_i}} $. Allora vale \[\sum_{i = 1}^{n}c_i^2 \leq \norm{v}^2\]
\end{thm}

\begin{thm}[ortogonalizzazione di Gram-Schmidt]
	Sia $ V $ uno spazio vettoriale con prodotto scalare definito positivo. Siano $ v_1, \ldots , v_n \in V $ vettori linearmente indipendenti. Possiamo allora trovare dei vettori $ u_1,\ldots, u_r $, con $ r \leq n $ ortogonali tra loro e tali che $ \forall i \leq r, \, Span\{v_1, \ldots, v_i\} = Span \{u_1, \ldots, u_i\} $. In particolare basterà procedere in modo induttivo e prendere
	\[\begin{cases}
	u_1 = v_1 \\
	u_{i} = v_i - \frac{\scprd{v_i}{u_1}}{\scprd{u_1}{u_1}}u_1 - \ldots - \frac{\scprd{v_i}{u_{i-1}}}{\scprd{u_{i-1}}{u_{i-1}}}u_{i-1}
	\end{cases}\]
\end{thm}

\begin{corollary}[esistenza della base ortonormale per prodotto scalare definito positivo]
	Dato $ V $ spazio vettoriale con prodotto scalare definito positivo esiste una base ortonormale di $ V $, ossia una base $ \{u_1,\ldots, u_r\} $ tale che $ \forall i \neq j \; \scprd{u_i}{u_j} = 0 $ e che $ \forall i \; \norm{u_i} = 1 $.
\end{corollary}

\begin{definition}[ortogonale e radicale]
	Sia $ V $ uno spazio vettoriale dotato di prodotto scalare $ \varphi $ e $ W \subseteq V $. Definiamo ortogonale di $ W $ l'insieme $ W^{\perp} = \{v \in V : \forall w \in W, \varscprd{v}{w} = 0\} $.\\
	Definiamo radicale di $ V $ l'insieme $ \operatorname{Rad}{(\varphi)} = \{v \in V : \forall w \in V, \scprd{v}{w} = 0\} $. Per definizione si ha che $ \operatorname{Rad}{(\varphi)} = V^{\perp} $
\end{definition}

\begin{propriety}[dell'ortogonale] \textbf{*}
	Sia $ V $ un $ \K $-spazio vettoriale dotato di prodotto scalare $ \varphi $ e siano $ U, W $ due sottospazi vettoriali di $ V $. Allora:
	\begin{enumerate}[label = (\roman*)]
		\item $ W \subseteq (W^{\perp})^{\perp} $ e se $ \varphi $ è non degenere $ W = (W^{\perp})^{\perp} $;
		\item $ W^{\perp} \cap U^{\perp} = (W + U)^{\perp} $;
		\item $ (W \cap U)^{\perp} \supseteq W^{\perp} + U^{\perp} $ e se $ \varphi $ è non degenere $ (W \cap U)^{\perp} = W^{\perp} + U^{\perp} $.
	\end{enumerate}
\end{propriety}

\begin{thm}
	Sia $ V $ un $ \K $-spazio vettoriale dotato di prodotto scalare $ \varphi $. Sia $ W $ un sottospazio di $ V $ tale che la restrizione del prodotto scalare $ \varphi\lvert_{W} $ sia non degenere. Allora \[V = W \oplus W^{\perp}.\] In particolare l'enunciato vale se il prodotto scalare definito positivo. 
\end{thm}

\begin{definition}[prodotto hermitiano]
	Sia $ V $ uno spazio vettoriale sul campo $ \C $. Un prodotto hermitiano su $ V $ è una funzione \[\scprd{\;}{\,} \colon V \times V \to \C \] che soddisfa le seguenti proprietà
	\begin{enumerate}[label=(\roman*)]
		\item $ \forall v,w \in V $ vale $ \scprd{v}{w} = \overline{\scprd{w}{v}} $ (coniugato)
		\item $ \forall v, w, u \in V $ vale $ \scprd{v}{w + u} = \scprd{v}{w} + \scprd{v}{u} $ e $ \scprd{v + w}{u} = \scprd{v}{u} + \scprd{w}{u} $
		\item $ \forall v, w \in V, \forall \lambda \in \C $ vale $ \scprd{\lambda v}{w} = \lambda \scprd{v}{w} $ e $ \scprd{v}{\lambda w} = \overline{\lambda} \scprd{v}{w} $
	\end{enumerate}	
\end{definition}

\begin{definition}[prodotto hermitiano standard]
	Dati due vettori colonna $ v, w \in \C^n $ definiamo il prodotto hermitiano standard come 
	\[v \cdot w = 
	\begin{pmatrix}
	\alpha_1 \\
	\vdots \\
	\alpha_n \\ 
	\end{pmatrix}
	\cdot 
	\begin{pmatrix}
	\beta_1 \\
	\vdots \\
	\beta_n \\ 
	\end{pmatrix}
	= \alpha_1 \overline{\beta_1} + \ldots + \alpha_n \overline{\beta_n}\]
\end{definition}

\begin{thm}[esistenza della base ortogonale per prodotto scalare generico] \label{thm:base_ortogonale}
	Sia $ V $ un $ \K $-spazio vettoriale non banale di dimensione finita dotato di un prodotto scalare. Allora $ V $ ha una base ortogonale. 
\end{thm}

\begin{thm}[Algoritmo di Lagrange] \textbf{*}
	Sia $ V $ un $ \K $-spazio vettoriale, $ \scprd{\;}{\,} $ un prodotto scalare se $ V $ e $ \mathscr{B} = \{v_1, \ldots, v_n\} $ una base qualunque di $ V $. 
	\begin{enumerate}
		\item Se $ v_1 $ non è isotropo, cioè $ \scprd{v_1}{v_1} \neq 0 $, poniamo
		\begin{align*}
		v'_1 & = v_1 \\
		v'_2 & = v_2 - \frac{\scprd{v_2}{v'_1}}{\scprd{v_1}{v_1}} v'_1 \\
		\vdots & \qquad \qquad  \vdots \\
		v'_n & = v_n - \frac{\scprd{v_n}{v'_1}}{\scprd{v_n}{v_1}} v'_1
		\end{align*}
		Così $ \scprd{v'_j}{v'_1} = 0 $ per ogni $ j \in {2, \ldots, n} $ e $ \mathscr{B}' $ è una base di $ V $.
		\item Se $ v_1 $ è isotropo, cioè $ \scprd{v_1}{v_1} \neq 0 $ allora 
		\begin{enumerate}
			\item Se $ \exists j \in {2, \ldots, n} $ tale che $ \scprd{v_j}{v_j} \neq 0 $ permuto la base $ \mathscr{B} $ in modo che $ v_j $ sia il primo vettore e procedo come in 1.
			\item Se $ \forall j \in {1, \ldots, n}, \ \scprd{v_j}{v_j} = 0 $ allora ci sono due casi
			\begin{itemize}
				\item $  \scprd{\;}{\,} $ è il prodotto scalare nullo e quindi ogni base è ortogonale
				\item $ \exists i \neq j :  \scprd{v_i}{v_j} \neq 0 $ in tale caso $ \scprd{v_i + v_j}{v_i + v_j} = 2 \scprd{v_j}{v_j} \neq 0 $. Scelgo allora una base di $ V $ in cui $ v_i + v_j $ è il primo vettore e procedo come in 1.
			\end{itemize}
		\end{enumerate}
	\end{enumerate}
	Dopo aver ortogonalizzato i vettori rispetto al primo, itero il procedimento su $ \{v'_2, \ldots, v'_n\} $ e così via. Alla fine ottengo una base ortogonale rispetto a $ \scprd{\;}{\,} $. Dunque vale l'enunciato del Teorema \ref{thm:base_ortogonale}.
\end{thm}

\begin{prop}[base ortonormale]
	Sia $ \{w_1, \ldots, w_n\} $ una base ortogonale di $ V $ con un prodotto scalare. Posto
	\[v_i = 
	\begin{cases*}
	\frac{w_i}{\sqrt{\scprd{w_i}{w_i}}} & se $ \scprd{w_i}{w_i} > 0 $ \\
	w_i & se $ \scprd{w_i}{w_i} = 0 $ \\
	\frac{w_i}{\sqrt{-\scprd{w_i}{w_i}}} & se $ \scprd{w_i}{w_i} < 0 $ \\
	\end{cases*}\]
	l'insieme $ \{v_1, \ldots, v_n\} $ è una base ortonormale di $ V $. 
\end{prop}

\begin{thm}[corrispondenza matrice - prodotto scalare]
	Sia $ V $ un $ \K $-spazio vettoriale e sia $ \mathscr{B} = \{v_1, \ldots, v_n\} $ base di $ V $. Dato un prodotto scale $ \varphi \colon V \times V \to \K $ la matrice del prodotto scalare è \[M_{\mathscr{B}}(\varphi) = (\varphi(v_i, v_j))_{\substack{i = 1, \ldots, n \\ j = 1, \ldots, n}}.\] Viceversa data $ M $ matrice simmetrica del prodotto scalare e $ u, w $ vettori di vettori colonna $ [u]_{\mathscr{B}} $ e $ [w]_{\mathscr{B}} $ si ha \[\varphi(v, w) = [u]_{B}^t \cdot M \cdot [w]_{\mathscr{B}}.\]
\end{thm}

\begin{prop}[radicale]
	Sia $ V $ un $ \K $-spazio vettoriale e $ \varphi $ un prodotto scalare su $ V $. Vale che $ \operatorname{Rad}{(\varphi)} = \{v \in V : \forall w \in V, \varphi(v, w) = 0\} = \{v \in V : M_{\mathscr{B}}(\varphi) \cdot [v]_{\mathscr{B}} = 0\} $.\\
	(Moralmente $ \operatorname{Rad}{(\varphi)} = \ker{M_{\mathscr{B}}(\varphi)} $). 
\end{prop}

\begin{definition}[spazio duale, funzionali]
	Sia $ V $ un $ \K $-spazio vettoriale. Si definisce spazio duale di $ V $ l'insieme delle applicazioni lineari da $ V $ in $ \K $ \[V^{*} = \mathscr{L}(V, \K) = \mathscr{L}(V) = \{L \colon V \to \K : L \text{ è lineare}\}.\] I suoi elementi vengono detti funzionali lineari da $ V $ in $ \K $ e risulta $ \dim{V} = \dim{V^{*}} $. 
\end{definition}

\begin{definition}[base duale]
	Sia $ V $ un $ \K $-spazio vettoriale. Fissata una base $ \{v_1, \ldots, v_n\} $ di $ V $ esiste una base $ \{\varphi_1, \ldots, \varphi_n\} $ di $ V^{*} $ ad essa associata detta base duale di $ v_1, \ldots, v_n $ definita come 
	\[\varphi_i (v_j) = \delta_{ij} =  
	\begin{cases*}
	1 & se $ i = j $ \\
	0 & se $ i \neq j $ \\
	\end{cases*}\]
\end{definition}

\begin{thm}[di rappresentazione] \label{thm:isom_duale}
	Sia $ V $ un $ \K $-spazio vettoriale di dimensione finita con un prodotto scalare non degenere. Allora l'applicazione
	\begin{align*}
	\Phi \colon V & \to V^{*} \\
	v & \mapsto \varphi_v = \scprd{v}{\cdot \,}
	\end{align*}
	dove $ \varphi_v $ è la funzionale tale che $ \forall w \in V : \varphi_v(w) = \scprd{v}{w} $, è un isomorfismo tra $ V $ e il suo duale $ V^{*} $. In altri termini, dato $ \varphi \in V^{*} $ esiste un unico $ v \in V $ tale che $ \forall w \in V, \varphi{(w)} = \scprd{v}{w} $ . In tale caso si dice che $ \varphi $ è rappresentabile.
\end{thm}

\begin{thm}[annullatore]
	Sia $ V $ un $ \K $-spazio vettoriale di $ \dim{V} = n $ e sia $ W $ sottospazio di $ V $. Sia inoltre $ \operatorname{Ann}{W} = \{\varphi \in V^{*} : \forall w \in W, \varphi(w) = 0\} $ l'annullatore di $ W $. Allora $ \operatorname{Ann}{W} $ è sottospazio di $ V^{*} $ e vale che $ \dim{(\operatorname{Ann}{W})} = n - \dim{W} $. 
\end{thm}

\begin{corollary}
	Sia $ V $ un $ \K $-spazio vettoriale di $ \dim{V} = n $ con prodotto scalare non degenere, sia $ W $ sottospazio di $ V $ e $ W^{\perp} $ il suo ortogonale. Siano inoltre $ V^{*} $ il duale di $ V $ e $ \operatorname{Ann}{W} $ l'annullatore di $ W $. Allora $ \Phi(W^{\perp}) = \operatorname{Ann}{W} $ (in altre parole $ \Phi|_{W^{\perp}} \colon W^{\perp} \to \operatorname{Ann}{W} $ è un isomorfismo) e vale quindi \[\dim{W} + \dim{W^{\perp}} = \dim{V}.\]
\end{corollary}

\begin{thm} \textbf{*}
	Sia $ V $ un $ \K $-spazio vettoriale dotato di un prodotto scalare $ \varphi $. Sia $ W $ sottospazio di $ V $, $ W^{\perp} $ il suo ortogonale e $ \operatorname{Rad}{(\varphi)} $ il radicale di $ \varphi $. Allora vale \[\dim{W} + \dim{W^{\perp}} = \dim{V} + \dim{(W \cap \operatorname{Rad}{(\varphi)})}\]
\end{thm}

\begin{thm}
	Sia $ \Phi \colon V \to V^{*} $ l'isomorfismo del Teorema \ref{thm:isom_duale}. Allora $ \operatorname{Rad}{(\varphi)} = \ker{\Phi} $ e $ M_{\mathscr{B^{*}}}^{\mathscr{B}} (\Phi) = M_{\mathscr{B}}(\varphi_v) $ dove $ \mathscr{B}^{*} $ è la base del duale associata alla base $ \mathscr{B} $ di $ V $. 
\end{thm}

\begin{corollary}
	Un prodotto scalare $ \varphi $ su $ V $ è non degenere $ \iff $ $ \operatorname{Rad}{(\varphi)} = \{O_V\} $ $ \iff $ $ \ker{M_{\mathscr{B}}(\varphi)} = O_V $ $ \iff $ $ M_{\mathscr{B}}(\varphi) $ ha rango $ n = \dim{V} $. 
\end{corollary}

\begin{corollary}
	Se $ \mathscr{B} $ è una base ortonormale allora la matrice del prodotto scalare è diagonale. Detti $ \lambda_i = \varphi(v_i, v_i) $ si ha 
	\[M_{\mathscr{B}}(\varphi) = 
	\begin{pmatrix}
	\lambda_1 & \cdots & 0 \\
	\vdots & \ddots & \vdots \\
	0 & \ldots & \lambda_n \\
	\end{pmatrix}\]
\end{corollary}

\begin{prop}[indice di nullità]
	Se $ \mathscr{B} = \{v_1, \ldots, v_n\} $ è una base ortonormale di $ V $ rispetto a $ \varphi $ prodotto scalare su $ V $, allora \[n_0 (\varphi) = \mathrm{card}\{i : \lambda_i = \varphi(v_i, v_i) = 0\} = \dim{\operatorname{Rad}{(\varphi)}}.\] Tale valore viene detto indice di nullità del prodotto scalare. 
\end{prop}

\begin{thm}[di Sylvester, indice di positività e di negatività] \label{thm:Sylvester}
	Sia $ V $ uno spazio vettoriale su $ \R $, $ \varphi $ un prodotto scalare su $ V $ e $ \mathscr{B} = \{v_1, \ldots, v_n\} $ una base di $ V $. Esiste un numero intero $ r = n_{+} (\varphi) $ che dipende solo da $ \varphi $ e non dalla base $ \mathscr{B} $, detto indice di positività, tale che ci sono esattamente $ r $ indici $ i $ tali che $ \varphi(v_i, v_i) = 1 $. Analogamente esiste un $ r' = n_{-} (\varphi) $ che dipende solo da $ \varphi $ e non dalla base $ \mathscr{B} $, detto indice di negatività, tale che ci sono esattamente $ r' $ indici $ i $ tali che $ \varphi(v_i, v_i) = - 1 $.
\end{thm}

\begin{definition}[segnatura e forma canonica dei prodotti scalari]
	Si definisce segnatura di un prodotto scalare $ \varphi $ su $ V $ spazio vettoriale su $ \R $ la terna $ (n_0 (\varphi), n_{+} (\varphi), n_{-} (\varphi)) $. Detta $ n = \dim{V} $ vale \[n_0 (\varphi) + n_{+} (\varphi) + n_{-} (\varphi) = n.\] Dato inoltre $ W $ sottospazio vettoriale di $ V $ valgono le seguenti caratterizzazioni
	\begin{itemize}
		\item $ n_0 (\varphi) = \dim{\operatorname{Rad}{(\varphi)}} $;
		\item $ n_{+} (\varphi) = \max{\{\dim{W} : W \subseteq V \text{ e } \varphi|_{W} > 0\}} $;
		\item $ n_{-} (\varphi) = \max{\{\dim{W} : W \subseteq V \text{ e } \varphi|_{W} < 0\}} $.
	\end{itemize}
	Per il Teorema \ref{thm:Sylvester} esiste una base ortonormale $ \mathscr{B} $ rispetto a $ \varphi $ in cui la matrice del prodotto scalare è della forma
	\[M_{\mathscr{B}}(\varphi) =
	\setlength{\arraycolsep}{0pt}
	\begin{pmatrix}
	\fbox{$ \Id_{n_{+}} $} & & \\
	& \fbox{$ -\Id_{n_{-}} $} & \\
	& & \fbox{$ 0_{n_{0}} $}
	\end{pmatrix} \]
	dove $ \Id_{r} $ è la matrice identità di dimensione $ r \times r $ e $ 0_p $ è la matrice nulla di dimensione $ p \times p $. \\
	Operativamente: per trovare la segnatura scrivo la matrice del prodotto scalare in una base ortonormale rispetto al prodotto scalare $ \varphi $; tale matrice è diagonale e la segnatura si legge sugli elementi della diagonale (ci sono $ n_{+} $ elementi uguali a 1, $ n_{-} $ elementi uguali a -1 e $ n_{0} $ elementi nulli). Per il Teorema Spettrale, posso semplicemente trovare gli autovalori della matrice del prodotto scalare (\textbf{Ci vanno altre ipotesi??})
\end{definition}

% CONFUSIONE SULLA NOTAZIONE PER IL PRODOTTO SCALARE, DA AGGIUSTARE

\clearpage

\section{Teorema spettrale, aggiunzione, operatori ortogonali e unitari}
% TEOREMA SPETTRALE, AGGIUNZIONE, OPERATORI ORTOGONALI E UNITARI

Sia $ V $ uno spazio vettoriale sul campo $ \K = \R \text{ o } \C $ di dimensione finita dotato di prodotto rispettivamente scalare o hermitiano definito positivo $ \scprd{\;}{\,} $.

\begin{thm}[endomorfismo aggiunto e matrice aggiunta]
	Dato $ T \colon V \to V $ endomorfismo esiste un unico endomorfismo $ T^{*} \colon V \to V $ tale che \[\forall u, v \in V, \ \scprd{Tu}{v} = \scprd{u}{T^{*}v}.\] Tale $ T^{*} $ viene detto endomorfismo aggiunto di $ T $. \\
	In termini di matrici, se $ \mathscr{B} $ è una base ortonormale di $ V $ e $ [T]_{\mathscr{B}}^{\mathscr{B}} $ è la matrice di $ T $ rispetto a tale base, allora la matrice di $ T^{*} $ rispetto alla stessa base è $ [T^{*}]_{\mathscr{B}}^{\mathscr{B}} = \overline{([T]_{\mathscr{B}}^{\mathscr{B}})^{t}} $ (trasposta coniugata). Se $ A $ è una matrice quadrata a coefficienti in $ \R $ o $ \C $, si definisce matrice aggiunta di $ A $ la matrice $ \overline{A^t} $.
\end{thm}

\begin{prop}[matrice dell'aggiunto]
	Sia $ T \in \End{(V)} $, $ \mathscr{B} $ una base di $ V $ e $ M $ la matrice del prodotto scalare scritta in tale base. Allora 
	\begin{itemize}
		\item $ \K = \R $: $ [T^{*}]_{\mathscr{B}}^{\mathscr{B}} = M^{-1} \, ([T]_{\mathscr{B}}^{\mathscr{B}})^{t} \, M $.
		\item $ \K = \C $: $ [T^{*}]_{\mathscr{B}}^{\mathscr{B}} = \overline{M^{-1}} \, \overline{([T]_{\mathscr{B}}^{\mathscr{B}})^{t}} \, \overline{M} $.
	\end{itemize}
\end{prop}

\begin{prop}[proprietà dell'aggiunzione]
	Dati $ T, S \in \End{(V)} $ vale
	\begin{enumerate}[label = (\roman*)]
		\item $ (T + S)^{*} = T^{*} + S^{*} $
		\item $ (T \, S)^{*} = S^{*} \, T^{*} $
		\item $ \forall \alpha \in \K, \ (\alpha \, T)^{*} = \overline{\alpha} \, T^{*} $
		\item $ (T^{*})^{*} = T $
		\item \textbf{*} $ \ker{T^{*}} = (\im{T})^{\perp} $ e $ \ker{T} = (\im{T^{*}})^{\perp} $;
		\item \textbf{*} $ \im{T^{*}} = (\ker{T})^{\perp} $ e $ \im{T} = (\ker{T^{*}})^{\perp} $.
	\end{enumerate}
\end{prop}

\begin{definition}[endomorfismo normale]
	$ T \in \End{(V)} $ si dice normale se commuta con il suo aggiunto, ovvero se $ T \; T^{*} = T^{*} \; T $.
\end{definition}

\begin{definition}[endomorfismo autoaggiunto]
	$ T \in \End{(V)} $ si dice autoaggiunto se $ T = T^{*} $. In termini di matrici se $ [T]_{\mathscr{B}}^{\mathscr{B}} $ è la matrice di $ T $ rispetto a una base ortonormale $ \mathscr{B} $ di $ V $ si ha
	\begin{itemize}
		\item $ \K = \R $: $ T $ è autoaggiunta $ \iff $ $ [T]_{\mathscr{B}}^{\mathscr{B}} $ è simmetrica (uguale alla trasposta).
		\item $ \K = \C $: $ T $ è autoaggiunta $ \iff $ $ [T]_{\mathscr{B}}^{\mathscr{B}} $ è hermitiana (uguale alla trasposta coniugata).
	\end{itemize}
\end{definition}

\begin{thm}
	Sia $ T \in \End{(V)} $ autoaggiunto. Se $ \lambda $ autovalore per $ T $, allora $ \lambda \in \R $. 
\end{thm}

\begin{thm}
	$ \K = \R $ e $ T \in \End{(V)} $ autoaggiunto. Allora
	\begin{enumerate}[label = (\roman*)]
		\item il polinomio caratteristico $ p_T(t) $ si fattorizza completamente e ha tutte le radici reali;
		\item $ T $ ha almeno un autovalore;
		\item se $ \{v_1, \ldots, v_r\} $ è un insieme di autovalori a due a due distinti allora $ v_1, \ldots, v_r $ sono a due a due ortogonali.
	\end{enumerate}
\end{thm}

\begin{prop}
	Sia $ T \in \End{(V)} $ e $ W $ sottospazio di $ V $ $ T $-invariante, i.e. $ T(W) \subseteq W $. Allora l'ortogonale $ W^{\perp} $ è $ T^{*} $-invariante.
\end{prop}

\begin{prop}
	Sia $ T \in \End{(V)} $ autoaggiunto e $ W $ sottospazio di $ V $ $ T $-invariante, i.e. $ T(W) \subseteq W $. Allora $ T\lvert_{W} $ è ancora autoaggiunto.
\end{prop}

\begin{thm}[Spettrale $ \K = \R $]
	Sia $ T \colon V \to V $ endomorfismo autoaggiunto se e solo se esiste una base ortonormale di $ V $ di autovettori per $ T $. \\
	In altri termini, ogni matrice simmetrica reale è simile a una matrice diagonale tramite una matrice ortogonale. In formule se $ S \in \mathrm{Mat}_{n \times n}(\R) $ è una matrice simmetrica reale esistono una matrice ortogonale $ O $ (i.e. $ O^{t} O = \Id $) rispetto al prodotto scalare standard di $ \R^{n} $ e una matrice diagonale $ D $ tali che \[D = O^{-1} S O = O^{t} S O.\]
\end{thm}

\begin{thm}[Spettrale $ \K = \C $] \textbf{*}
	Sia $ T \colon V \to V $ endomorfismo normale se e solo se esiste una base ortonormale di $ V $ di autovettori per $ T $.\\
	In altri termini, ogni matrice normale è simile a una matrice diagonale tramite una matrice unitaria. In formule se $ N \in \mathrm{Mat}_{n \times n}(\C) $ è una matrice normale esiste esistono una matrice unitaria $ U $ (i.e. $ \overline{U^t} U = \Id $) rispetto al prodotto hermitiano standard di $ \C^{n} $ e una matrice diagonale $ D $ tali che \[D = U^{-1} N U = \overline{U^t} N U.\]
\end{thm}

\begin{thm}[Spettrale $ \K = \C $ per endomorfismi autoaggiunti]
	Sia $ T \colon V \to V $ endomorfismo normale allora se esiste una base ortonormale di $ V $ di autovettori per $ T $ (una sola implicazione).\\
	In altri termini, ogni matrice hermitiana è simile a una matrice diagonale tramite una matrice unitaria. In formule se $ H \in \mathrm{Mat}_{n \times n}(\C) $ è una matrice hermitiana esiste esistono una matrice unitaria $ U $ (i.e. $ \overline{U^t} U = \Id $) rispetto al prodotto hermitiano standard di $ \C^{n} $ e una matrice diagonale $ D $ tali che \[D = U^{-1} H U = \overline{U^t} H U.\]
\end{thm}

\begin{prop}
	Sia $ T \in \End{(V)} $ (autoaggiunto se $ \K = \R $). Allora \[T = O_{\End{V}} \iff \forall v \in V, \ \scprd{Tv}{v} = 0\]
\end{prop}

\begin{thm}
	Sia $ V $ uno spazio vettoriale su $ \R $ con prodotto scalare definito positivo. Sia $ T \in \End{(V)} $ tale che $ T T^{*} = T^{*} T $. Allora $ V = \ker{T} \oplus \im{T} $. 
\end{thm}

\begin{definition}[endomorfismo ortogonale e matrice ortogonale]
	$ \K = \R $. Un endomorfismo $ U \colon V \to V $ tale che $ U^{*} = U^{-1} $ si dice endomorfismo ortogonale. In termini di matrici se $ \mathscr{B} $ è un base ortonormale, $ [U]_{\mathscr{B}}^{\mathscr{B}} $ è ortogonale $ \iff $ $ [U^{-1}]_{\mathscr{B}}^{\mathscr{B}} = ([U]_{\mathscr{B}}^{\mathscr{B}})^t $.\\
	Equivalentemente $ M \in \mathrm{Mat}_{n \times n}(\R) $ si dice ortogonale se $ M^{t} = M^{-1} $.
\end{definition}

\begin{prop}
	Le seguenti affermazioni sono equivalenti.
	\begin{enumerate}[label = (\roman*)]
		\item $ A \in \mathrm{Mat}_{n \times n}(\R) $ è ortogonale.
		\item Le righe di $ A $ sono vettori ortonormali rispetto al prodotto scalare standard.
		\item Le colonne di $ A $ sono vettori ortonormali rispetto al prodotto scalare standard.
	\end{enumerate}
\end{prop}

\begin{definition}[endomorfismo unitario e matrice unitaria]
	$ \K = \C $. Un endomorfismo \linebreak $ U \colon V \to V $ tale che $ U^{*} = U^{-1} $ si dice endomorfismo unitario. In termini di matrici se $ \mathscr{B} $ è un base ortonormale, $ [U]_{\mathscr{B}}^{\mathscr{B}} $ è unitaria $ \iff $ $ [U^{-1}]_{\mathscr{B}}^{\mathscr{B}} = \overline{([U]_{\mathscr{B}}^{\mathscr{B}})^{t}} $\\
	Equivalentemente $ M \in \mathrm{Mat}_{n \times n}(\C) $ si dice unitaria se $ \overline{M^{t}} = M^{-1} $.
\end{definition}

\begin{prop}
	Le seguenti affermazioni sono equivalenti.
	\begin{enumerate}[label = (\roman*)]
		\item $ A \in \mathrm{Mat}_{n \times n}(\C) $ è unitaria.
		\item Le righe di $ A $ sono vettori ortonormali rispetto al prodotto hermitiano standard.
		\item Le colonne di $ A $ sono vettori ortonormali rispetto al prodotto hermitiano standard.
	\end{enumerate}
\end{prop}

\begin{thm}
	Sia $ U \in \End{(V)} $ tale che $ U^{*} = U^{-1} $ (i.e. ortogonale o unitario). Se $ \lambda $ è autovalore per $ U $ allora 
	\begin{enumerate}[label = (\roman*)]
		\item $ \abs{\lambda} = 1 $ (in particolare se $ \lambda \in \R $ allora $ \lambda  = \pm 1 $);
		\item Se $ U v = \lambda v $ allora $ U^{*} v = \overline{\lambda} v $ (in particolare $ \overline{\lambda} $ è autovalore di $ U^{*} = U^{-1} $ rispetto allo stesso autovettore). 
	\end{enumerate}
\end{thm}

\begin{thm}
	Dato $ U \in \End{(V)} $ sono equivalenti
	\begin{enumerate}[label = (\roman*)]
		\item $ U^{*} = U^{-1} $;
		\item $ \forall v, w \in V, \ \scprd{Uv}{Uw} = \scprd{v}{w} $ ($ U $ è un'isometria)
		\item $ \forall v \in V, \ \norm{Uv} = \norm{v} $.
	\end{enumerate}
\end{thm}

\begin{prop}
	Sia $ U \in \End{(V)} $ ortogonale o unitaria e sia $ W $ un sottospazio di $ V $ $ U $-invariante. Allora $ W^{\perp} $ è $ U $-invariante.
\end{prop}

\begin{thm}[Spettrale per gli endomorfismi unitari]
	$ \K = \C $. Sia $ U \in \End{(V)} $ unitario. Allora esiste una base ortonormale di $ V $ di autovettori per $ U $.
\end{thm}

\begin{thm}[forma canonica degli endomorfismi ortogonali]
	$ \K = \R $. Sia $ Q \colon V \to V $ un endomorfismo ortogonale. Allora esiste una base ortonormale $ \mathscr{B} $ di $ V $ in cui la matrice di $ Q $ ha la forma
	\[[Q]_{\mathscr{B}}^{\mathscr{B}} = 
	\setlength{\arraycolsep}{0pt}
	\begin{pmatrix}
	\fbox{$ \Id_{p} $} &  &  &  &  \\ 
	& \fbox{$ - \Id_{q} $} &  &  &  \\ 
	&  &\fbox{$ R_{\theta_1} $} &  &  \\ 
	&  &  & \ \ddots \ &  \\ 
	&  &  &  & \fbox{$ R_{\theta_k} $}
	\end{pmatrix} \]
	dove $ p, q, k \in \N $ con $ p + q + 2 k = n = \dim{V} $, $ \Id_r $ è la matrice identità di dimensioni $ r \times r $ e $ R_{\theta_i} $ è una matrice rotazione non banale $ 2 \times 2 $ 
	\[R_{\theta_i} = 
	\begin{pmatrix}
	\cos{\theta_i} & - \sin{\theta_i} \\
	\sin{\theta_i} & \cos{\theta_i}
	\end{pmatrix}
	\quad
	\text{ con $ \theta_{i} \neq 0, \pi $ ($ + 2 k \pi, \ \forall k \in \Z $)}\] 
\end{thm}

\begin{prop}
	\textsf{$ O $ matrice ortogonale $ \Rightarrow \det O = \pm 1 $.\\
		$ U $ matrice unitaria $ \Rightarrow \abs{\det O} = 1 $.}
\end{prop}

\begin{definition}
	\textsf{$ O(V) $ è il gruppo degli endomorfismi ortogonali con l'operazione di composizione. $ O(\R^n) = O(n) $.\\
		Il sottogruppo delle di $ O(V) $ con $ \det = +1 $ è il gruppo ortogonale speciale $ SO(V) $. }
\end{definition}

\begin{definition}
	\textsf{$ U(V) $ è il gruppo degli endomorfismi unitari con l'operazione di composizione. $ U(\C^n) = O(n) $.\\
		Il sottogruppo delle di $ U(V) $ con $ \det = +1 $ è il gruppo unitario speciale $ SU(V) $. }
\end{definition}

\clearpage

\section{Miscellanea}
% MISCELLANEA

\begin{thm}[forma canonica delle involuzioni] \textbf{*}
	Sia $ V $ un $ \K $-spazio vettoriale con $ \dim{V} = n $ e sia $ f \colon V \to V $ un'applicazione lineare tale che $ f^{2} = \Id $. Allora esiste una base $ \mathscr{B} $ di $ V $ tale che 
	\[[f]_{\mathscr{B}}^{\mathscr{B}} = 
	\setlength{\arraycolsep}{0pt}
	\begin{pmatrix}
	\fbox{$ \Id_k $} & \\
	& \fbox{$ -\Id_{n - k} $}
	\end{pmatrix}\]
	con $ k \in \N $ univocamente determinato.
\end{thm}

\begin{thm}[forma canonica delle proiezioni]
	Sia $ V $ un $ \K $-spazio vettoriale con $ \dim{V} = n $ e sia $ f \colon V \to V $ un'applicazione lineare tale che $ f^{2} = f $ (proiezione). Allora esiste una base $ \mathscr{B} $ di $ V $ tale che 
	\[[f]_{\mathscr{B}}^{\mathscr{B}} = 
	\setlength{\arraycolsep}{0pt}
	\begin{pmatrix}
	\fbox{$ \Id_k $} & \\
	& \fbox{$ 0_{n - k} $}
	\end{pmatrix}\]
	con $ k \in \N $ univocamente determinato.
\end{thm}

\begin{thm} \textbf{*}
	Sia $ V $ uno spazio vettoriale su $ \R $ di dimensione $ \geq 1 $ e sia $ f \colon V \to V $ un'applicazione lineare tale che $ f^{2} = - \Id $. Allora esiste una base $ \mathscr{B} $ di $ V $ tale che 
	\[[f]_{\mathscr{B}}^{\mathscr{B}} = 
	\setlength{\arraycolsep}{0pt}
	\begin{pmatrix}
	& \fbox{$ -\Id_m $}\\
	\fbox{$ \Id_m $} & 
	\end{pmatrix}\]
	con $ m \in \N $ univocamente determinato. In particolare tale base è $ \mathscr{B} = \{v_1, \ldots, v_m, f(v_1), \ldots, f(v_m)\} $.
\end{thm}

\begin{thm}[forma canonica delle matrici antisimmetriche reali] \textbf{*}
	Sia $ A \in \mathrm{Mat}_{n \times n}(\R) $ antisimmetrica. Allora esiste una matrice ortogonale $ M \in O(n) $ tale che 
	\[M^{-1} A M = M^{t} A M = 
	\setlength{\arraycolsep}{0pt}
	\begin{pmatrix}
	\fbox{$ H_{a_1} $} & & &\\
	& \ \ddots \ & & \\
	& & \fbox{$ H_{a_k} $} & \\
	& & & \fbox{$ 0 $}
	\end{pmatrix}
	\qquad
	\text{ con }
	H_{a_i} = 
	\setlength{\arraycolsep}{3pt}
	\begin{pmatrix}
	0 & a_{i} \\
	-a_{i} & 0
	\end{pmatrix}\]
\end{thm}


\begin{center}
	\textbf{TEST MERGE AXK}
\hspace{5cm}
	\textbf{TEST MERGE}
\end{center}

\end{document}
